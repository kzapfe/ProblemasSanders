\documentclass[letterpaper,10pt]{article}

\usepackage[utf8]{inputenc}
\usepackage{amsmath,amssymb}
\usepackage{graphicx}
\usepackage{caption}
\usepackage{subcaption}

%Nueva convenvion para las subfigures es el subcaption package

\usepackage[spanish]{babel}
\usepackage{bm}

\bibliographystyle{alpha}


\newcommand{\xfase}{\mathbf{x}}
\newcommand{\qfase}{\mathbf{q}}
\newcommand{\pfase}{\mathbf{p}}
\newcommand{\xifase}{ {\boldsymbol{\xi}} }
\newcommand{\mufase}{ {\boldsymbol{\mu}} }
\newcommand{\Ifase}{\mathbf{I}}
\newcommand{\Pfase}{\mathbf{P}}
\newcommand{\Scat}{\mathbf{S}}
\newcommand{\Jsimp}{\mathbf{J}}
\newcommand{\Dom}{\mathbb{D}}
\newcommand{\Var}{\mathbb{M}}
\newcommand{\bra}[1]{\langle #1|}
\newcommand{\ket}[1]{|#1\rangle}
\newcommand{\braket}[2]{\langle #1|#2\rangle}



\DeclareMathOperator*{\cod}{cod}
\DeclareMathOperator*{\traza}{traza}


\title{Notas Varias sobre problemas varios}
\author{ W. P. Karel Zapfe\\Fac. de Ciencias, UNAM}


\begin{document}

\section{Billares Cilindricos Truncados.}

\subsection{Coordenadas}
El billar de una partícula dentro de un cilíndro truncado
requiere un buen sistema de coordenadas. Aunque no tenemos claro
aún que tan caótico es el sistema en sí, tenemos experimentos numéricos
aceptables que parecen indicar  caos fuerte (ergodicidad, hiperbolicidad,
mezclante).

La anidación de conos es una forma usual
de buscar la hiperbolicidad en un sistema dinámico.
A grandes rasgos consiste en definir objetos que
coinciden con nuestra idea intuitiva de un cono, en el haz tangente
al espacio fase del sistema dinámico. Si este campo de conos es
tal que bajo el \emph{push-forward} la imagen de cada cono se anida dentro
del cono en el nuevo punto, el sistema es hiperbólico y además cuenta
con variedades hiperbólicas en casi cada punto lo suficientemente
largas como para encontrar ergodicidad.

Si en un momento nos interesara seguir esta linea de razonamiento
para demostrar caos fuerte en los billares, necesitamos
un sistema de coordenadas para el billar. Dado que que el argumento
de los conos funciona mejor en sistemas dinámicos discretos en el tiempo,
debemos tener una sección de Birkhoff-Poincaré adecuada.
Una malla ortogonal sobre el 
cilindro puede resultar cómoda para trabajar. 

El siguiente sistema de coordenadas funciona bien 
si el corte truncado es a cualquier
altura, pero es más claro explicarlo si el plano de corte no
interseca la base del cilindro. El origen estará en la base, al centro
del círculo, como un sistema polar plano común. Sin embargo,
por consistencia de conceptos y unidades, usaré en la dirección 
ortogonal al radio  \emph{longitud de arco} en lugar de ángulos.
Esto hace que su cubriente parezca un tríangulo, y que la única
coordenada en el origen sea $(0,0)$, a diferencia de las polares normales,
en los cuales colapsa todo el sistema. La gran desventaja de
esta convención es que las lineas radiales no son aquellas que tienen
la coordenada ortogonal constante. De ahora en adelante las lineas radiales
y su continuación definirán lo que denotaré por la coordenada $\rho$ y
las ortogonales a ésta serán la coordenada $s$ (figura \ref{suelo}).

En un segundo paso extendemos las lineas
radiales sobre la superficie del cilindro
paralelas al eje $z$. Éstas continuarán los valores de la coordenada
$\rho$ siguiendo la longitud de arco total desde el origen.
 La coordenada ortogonal $s$ serán los círculos perimetrales al cilindro,
paralelos siguiendo el eje $z$ como en la figura \ref{pared}. 

Finalmente, sobre el plano inclinado que termina el cilindro truncado,
comenzamos lo realmente molesto. Hay dos opciones. Ambas tienen
que tomar en cuenta que para que sean continuas, tenemos que extender
las lineas $\rho$ desde fuera hacia dentro. La superficie es una elípse,
asi que lo correcto sería usar alguna buena modificación de las coordenadas 
elípticas. Sin embargo, como lo que nos interesa es la longitud de arco, 
caemos en el problema de las integrales elípticas. Puede ser que haciendo
una buena reducción de ésto, el problema no quede tan espantoso.
Para evaluar numericamente el valor de las coordenadas en cada
punto habrá que escoger una rutina numérica adecuadamente rápida
(¿tal vez \verb1gsl_elliptic_n.h1 ?). No estoy muy seguro que el esfuerzo númerico
valga la pena, pero calcular la derivada del mapeo va a
requerrir al menos la manipulación de éstas formas.
Otra forma un poco más sucia es usar coordenadas polares sobre el plano
inclinado, pero eso producirá discontinuidades a lo largo
de las lineas $s$.


\begin{figure}
        \centering
        \begin{subfigure}[b]{0.3\textwidth}
                \centering
                \includegraphics[width=\textwidth]{CoordenadasBillar01.pdf}
                \caption{Suelo}
                \label{suelo}
        \end{subfigure}%
        ~ %add desired spacing between images, e. g. ~, \quad, \qquad etc.
          %(or a blank line to force the subfigure onto a new line)
        \begin{subfigure}[b]{0.3\textwidth}
                \centering
                \includegraphics[width=\textwidth]{CoordenadasBillar02.pdf}
                \caption{Pared}
                \label{pared}
        \end{subfigure}
        ~ %add desired spacing between images, e. g. ~, \quad, \qquad etc.
          %(or a blank line to force the subfigure onto a new line)
        \begin{subfigure}[b]{0.3\textwidth}
                \centering
                \includegraphics[width=\textwidth]{CoordenadasBillar03.pdf}
                \caption{Techo}
                \label{techo}
        \end{subfigure}
        \caption{Construccion de las coordenadas}\label{WireFrameCilindro}
\end{figure}



\section{Dos Discos en un círculo}

A partir del Phys.Rev. que mandaste, hay unos comentarios que hacer y
discutir. 
El articulo versa mayoritariamente sobre
 los conceptos de entropía y temperatura. 
Usan la definición de entropía de Boltzman,
\begin{equation}
  S=k \log (\Gamma(E,L) ),
\end{equation}
restringida a capas de momemto angular constante.
Por supuesto que asumen que el sistema es ergódico o al menos
tiene un componente ergódico tan grande que los demás no afectan. Lo
que a mi me gustaría ver sería nuestro concepto ``intuitivo''
de entropía y temperatura en este sistema:
\begin{equation}
T=\Big(\frac{\partial S}{\partial E} \Big)^{-1}.
\end{equation}
En un gas ideal esto es bastante claro: asociamos entropía a la falta
de un patrón claro, a una forma de desorden o a excesiva riqueza en la
estructura. Un gas ``muy caliente'' ya se encuentra sin patrón definido,
por lo tanto aumentando la energía total no va a cambiar mucho su
configuración, la derivada parcial es pequeña y la temperatura
es alta. Busquemos interpretar así la figura 1 del artículo.

Vemos que la temperatura es máxima para el caso de momento
angular total igual a cero. ¿Tiene esto sentido? 
$L=0$ significa que o las partículas giran
en direcciones opuestas o recorren diámetros del círculo. 
Por conservación de $L$ si se llegan a estrellar
sólo pueden cambiar entre estos dos estados. Si aumentamos la energía
no hay cambio en la calidad del movimiento (rescalando las velocidades
se ven iguales), asi que la temperatura es efectivamente alta en
el sentido de que no hay un cambio cualitativo en la dinámica
del estado. No se puede acceder a otra parte del espacio fase conforme
aumentamos la energía.

Si tenemos, en cambio momento angular muy grande, tenemos varios 
posibles escenarios. Las dos partículas pueden estar girando
en la misma dirección, con velocidades parecidas. Pueden
cambiar de esar dando vueltas muy cerca de la pared a cruzar cerca
del diámetro, pero cambiando la magnitud de su velocidad mucho. 
O incluso puede una estar girando en la dirección opuesta,
pero una de forma lenta en relación a la otra. Los choques pueden
hacer que cambie de un estado a otro. 
El volumen del espacio fase accesible parece ser mayor,
ya que tenemos tres forma cualitativas distintas de movimiento. 
La energía puede distribuirse \emph{aparentemente} de
forma más rica en este caso, por lo tanto el sistema esta
``frio'' y puede calentarse. 


\section{Animales circulares enfermos y contagiosos}

Ordenemos los cálculos hechos hasta ahora, para disponer de ellos
en cualquier momento. 
Los cuadrados habitables tienen longitud 2, centrados en el
origen de coordenadas. 
Los animales se llaman Uno y Dos, con subíndices 
en las coordenadas indicando su propiedad, y
ambos tienen radio $\epsilon$.
Los animales se mueven con el centro
dentro de su cuadrado de confianza
siguiendo una ruta de caminante aleatoreo.

\subsection{Caso simplificado}

Uno de los dos animales se mueve en una linea en el eje 
horizontal, de lado al lado. Aun así usaremos las cuatro
coordenadas, con el sobreentendido de que tenemos una
restricción absoluta.
Todas las condiciones se resumen en la siguiente
lista de ecuaciones.

\begin{align}
& x_1, y_1, x_2 \in [-1,1] \\
& y_2=0 \\
& (x_1-x_2)^2+(y_1-y_2)^2 \geq (2\epsilon)^2 \label{circle}
\end{align}

Para ahorrarnos talacha espantosa,
vamos a usar un sistema de coordenadas relativas
que preserve tamaño.

\begin{align} \label{transformas}
 \frac{1}{\sqrt{2}}(x_1-x_2)=x \text{ y } 
\frac{1}{\sqrt{2}}(y_1-y_2)=y, \\
\frac{1}{\sqrt{2}}(x_1+x_2)=X \text{ y } 
\frac{1}{\sqrt{2}}(y_1+y_2)=Y. 
\end{align}

La transformación \ref{transformas} es
una rotación de coordenadas por $pi/2$ en 
cada par de coordenadas de nomnbre $a_i$, digamos

\begin{equation}
  \begin{pmatrix}
    a \\
    A
  \end{pmatrix}
  =\frac{1}{\sqrt{2}}
  \begin{pmatrix}
    1 & -1 \\
    1 & 1
  \end{pmatrix}
  \begin{pmatrix}
    a_1 \\
    a_2
  \end{pmatrix}
\end{equation}

La condición \ref{circle} queda así:
\begin{equation}
x^2+y^2 \geq 2 \epsilon^2
\end{equation}

La condición 4 queda como $y=Y$.  De ahora
en adelante  se sobreentiende que este  la sección del
espacio en el que hacemos los razonamientos siguientes.

La frontera de absorción es un cilindro
contenido de radio $\sqrt{2} \epsilon$, que va de 
arista a arista opuesta en un cubo de lado $2$
(figura \ref{CilindroenCubo01}).

\begin{figure}
\centering
\includegraphics[width=0.75\textwidth]{CilindroCubo03.png}
\caption{El cilindro absorvente y el espacio fase total
del problema reducido. El cubo es
una frontera reflejante para la particula.}
\label{CilindroenCubo01}
\end{figure}

En este caso tenemos que solucionar una ecuación de
difusión para la partícula en un espacio de
tres dimensiones, con condiciones
reflejantes en las paredes del cubo y condiciones
absorventes en el cilindro. Nuestra intención es encontrar
un tiempo medio de primer encuentro con el cilindro
para condiciones iniciales arbitrarias en el cubo. 
Esto correspondería \emph{a grosso modo} al tiempo de
primer contacto y posible contagio.


\subsection{Los animales se mueven en el mismo habitat.}

Prescindimos de la condición $y_2=0$, pero mantenemos
el mismo sistema de coordenadas auxiliar. Ahora
sabemos como se ve la superficie absorvente
en una sección de tres dimensiones, la sección $y-Y=0$. 

La superficie absorvente sigue siendo un cilindro, pero
encajado esta vez en un hipercubo, aun de forma diagonal.
Las aristas son caras de dimensión dos. 
Este cilindro tiene un corte circular, en cada plano
$X,Y$ constantes. El cilindro se extiende en recto
en las direcciónes $X,Y$, asi que su corte en el plano
$x-X=0$ se ve igual también como la figura 
\ref{CilindroenCubo03}.






\end{document}


