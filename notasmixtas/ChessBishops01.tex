\documentclass[leterpaper, 12pt]{article}

\usepackage[utf8]{inputenc}
\usepackage{graphicx}

\title{About Chess Bishops, tacit assumptions and geometry}

\author{K. Zapfe}

\begin{document}

\maketitle

\abstract{
When we change coordinates on the problem of 2 $f$-dimensional random
walkers to represent its motion as a single random walker in 
a space of $2f$ dimensions we use \emph{tacitally} an extra
assumption wich changes the dynamics and the geometry of the space.
This could have nefarious consecuences on our results.
}

\section{Luca, Sebastian and David $1$-dimensional walkers}

In the article \cite{SandersLuca} two one dimensional walkers 
with an intersecting territorry where presented as equivalent
to a single 2d random walker on an unfolded domain. There
was an implicit assumption which was never adressed about the dinamyc 
of this representative system. Depending of the choice of
rules for the original pair of random wallkers, the representative walker
can or not cover the whole 2d space.

We have three different options, depicted in figure \ref{3casos}.

\begin{itemize}
\item Both 1d walkers move simultaneously at every time step.
\item Onle one of the walkers moves each time step, choosen at random.
\item Every time step it is  choosen at random to move one or two of the
  1d walkers.
\end{itemize}


\begin{figure}[h]
  \centering
  \includegraphics[width=0.75\textwidth]{3casosrw01.pdf}
  \caption{The three possible dynamics, as indicated previously.
  In each case, the RW can only go to the red squares in one
time step. In case ``a'', the ``chess' bishop'' case, the walker is
condemned to stay in the diagonals and explore just half of the
unfolded territory, as explained below. Cases ``b'' and ``c'' can
explore the whole space, but have different difussion constants. }
  \label{3casos}
\end{figure}

The first one is the most problematic. We shall call it
the bishop dynamics, as the representative
system can only move along diagonals. 
It can be seen that for
some configurations of the initial condition it is possible
that the two walkers do not met before a collition
with the boundary of their territory. We could say that when we change
represantation to the full face space, it is divided into white and
black squares, as a chess board, and the 2d walker
is forced to move along diagonals, as a bishop. Thus, if the
line which represents the colition condition belongs to 
a different set of squares, cannot be reached until a wall is hit,
then the walker moves on the other set of diagonals. On the unfolded
version of the space this would produce a very odd looking
target, consisting of two parallel lines on opposite realitations
of the fundamental domain. This problem cannot be approximated by
the circles used in the mentioned work, and a numerical calculation
shows that the first encounter times on average follow a very
different curve that the ones which adapt to Szabo's prediction
for an anular region. 


\begin{figure}[h]
  \centering
  \includegraphics[width=0.5\textwidth]{cuadritosajedrex01.pdf}
  \caption{The first case for the dynamics after unfolding
    along the red lines. If the walker, represented by the green
    circle, begins on the black squares, it can never ocupy the
    white ones, thus the absorving target is reduced only to 
    the deep blue lines. If it starts on the white ones, its target 
    are the light blue squares. }
  \label{bishop01}
\end{figure}

\section{ 2d walkers'}

Here it goes even WORSE.

\end{document}
