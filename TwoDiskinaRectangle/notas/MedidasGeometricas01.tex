%% Las medidas necesarias para obtener los tiempos medios
%% De colisiones usando resultados ergodicos.

\documentclass{article}

\usepackage{amsmath}

\begin{document}
%%Volumen 4d

El Volumen Libre es el volumen del hipercubo menos el volumen del
hipercilindro interior. El Volumen del cilindro de radio $\sqrt{2}r_0$
(el factor raiz es debido a la rotación.) 
encajado diagoanalmente en el hipercubo con lados $a, a, b, b$ está dado
por:
\begin{equation}
V_{4cil}=r_0^2 [4 \pi a b + r_0^2 (10\pi -13/3) -4 \sqrt{2} \pi r_0 (a+b) ]
\end{equation}

%%Recuerdas cual era la parte debida a las cuñas ????

El área esta dada por la siguiente patada en los huevos:

\begin{equation}
A_{4cil}=8\pi (a-\sqrt{2}r_0)(b-\sqrt{2}r_0)+r_0^3(\pi/2-3/2).
\end{equation}

En el caso del billar rígido, el área efectiva de movimiento
está afectada por el tamaño de los discos, asi que las siguientes
substituciones son pertinentes:

\begin{align}
a &\rightarrow w-2r_0 \\
b &\rightarrow l-2r_0.
\end{align}

Esto, por supuesto, hace que la expresiones anteriores sean aún más desgradables.

\begin{equation}
V_{4cil}=r_0^2 [4 \pi (w-\sqrt{2}r_0)(l-\sqrt{2}r_0) + r_0^2 (10\pi -13/3) -
4 \sqrt{2} \pi r_0 (l+w-4r_0) ]
\end{equation}

\begin{equation}
A_{4cil}=8\pi (l-r_0(\sqrt{2}+2))(w-r_0(\sqrt{2}-2))+r_0^3(\pi/2-3/2).
\end{equation}

El Volumen Libre es simplemente $a^2b^2$ menos lo de arriba. 

\begin{multline}
V_{libre}=(l-\sqrt{2}r_0)^2(w-\sqrt{2}r_0)^2- \\
r_0^2 [4 \pi (w-\sqrt{2}r_0)(l-\sqrt{2}r_0) + r_0^2 (10\pi -13/3) -
4 \sqrt{2} \pi r_0 (l+w-4r_0) ]
\end{multline}

Para usar la expresión ergódica, no se que tan conveniente sea simplificar esta
payasada.

\end{document}
