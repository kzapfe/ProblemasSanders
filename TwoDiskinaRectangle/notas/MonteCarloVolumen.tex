\documentclass[a4paper,10pt]{article}


%\usepackage[spanish]{babel}
\usepackage[utf8]{inputenc}
\usepackage{amsmath, amsfonts}
\usepackage{graphicx}
\usepackage{natbib}
\usepackage{caption}
\usepackage{subcaption}



\bibliographystyle{alpha}

%opening
\title{Monte-Carling the Volume}
\author{ W. P. K. Zapfe}
%\address{Departamento de Física, Facultad de Ciencias, Universidad Nacional Autónoma de México, Ciudad Universitaria, Del.~Coyoacán, México D.F. 04510, Mexico}

\usepackage{mathptmx}

\newcommand{\defeq}{:=}
\newcommand{\mean}[1]{\left \langle #1 \right \rangle}
\newcommand{\rd}{\, \mathrm{d}}
\newcommand{\RR}{\mathbb{R}}
\newcommand{\vv}{\mathbf{v}}
\newcommand{\indicator}[1]{\mathbf{1}_{ \{   #1 \} } } 
\newcommand{\Indi}{\mathbf{1}}


\setlength{\parskip}{10pt}
\setlength{\parindent}{0pt}

\begin{document}

\maketitle

\section{Volume}

We shall test the formula for the unavailable volume shown in the 
previous report.  To remind, the volume \emph{unavailable} for
the initial conditions in the 4 dimensional configuration space was 
given in the following form:
\begin{equation}
V=4 \pi abd^2-\frac{8 d^3}{3}(a+b)+\frac{d^4}{2},
\end{equation} 
This formula applies only if $d< 2a , 2b $.
The geometric abbreviations are as following, $a=w/2-r$, $b=h/2-r$
and $d=2 r$, where $w,h$ are the width and height of the billiard table,
$r$ is the radius of the disks. I shall rewrite the above formula
in explicit dependence of $r$:
\begin{equation}
V(r)=r^4 (128/3+8+16\pi)-r^3(w+l)(32/3+8\pi)+r^2 w l 4 \pi 
\end{equation}

For a simple test we run an homogeneous random conditions generator
on the $2a\times 2a\times  2a \times 2b\times 2b$ rectangle. This of course
is more efficient than searching on the whole billiard table, but needs a correction.
As I count every time that I hit inside the prohibited volume, that is,
every time that the centres of the cylinder are at distance less than $d$,
I am counting the fraction of volume occupied by the cylinder inside
the effective hyper-cube. This hyper-cube has volume $(w-2r)^2 (l-2r)^2$.
The function which shall fit the data is then $V(r)/((w-2r)^2 (l-2r)^2)$.
The Monte Carlo routine uses 4 million tries, and, as the graphic below shows,
this becomes terrible inefficient for the bigger radius end, but fits perfectly for most
of the range. We take into account that the maximal radius possible
is, for $w=l=1$, $r=1-\sqrt{1/2}\approx 0.29$. The Monte Carlo shows inefficacy at
$r\approx 0.27$. Notice that is inefficiency, not inaccuracy: the counting falls
more often than not inside of the cylinder\ldots.


\begin{figure}
\centering
\includegraphics[width=0.8\textwidth]{VolumenOccupiedByCilinder01.pdf}
\caption{Fraction of occupied volume by the 
Four Dimensional Prohibition Cylinder for the configuration space of the 
Two Disks problems.}
\end{figure}

\begin{figure}
\centering
\includegraphics[width=0.8\textwidth]{Muestrax1x2Subspace01.pdf}
\caption{ A projection of the initial conditions for the $r=0.1519$ case, in the
$(x_1,x_2)$ subspace. It is clear how the density in the diagonal band decreases
because of the exclusion condition. }
\end{figure}



\end{document}


