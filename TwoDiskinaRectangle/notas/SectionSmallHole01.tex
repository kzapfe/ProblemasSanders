\subsection{Area of small hole in a wall}
\pmb{Check this with David, maybe sent it to the freezer}

We can also consider the impact with a very small portion of any of the
walls. If the disks are small enough, this would correspond to an escape time. 
Let us assume a hole of size $\epsilon+2r$ for suitable values of $r$ in
the right side of the wall. Let us assume that the hole is
centred, then the expression to evaluate is
\begin{equation}\label{areaindicagujero}
 A_{x_1+} =  \int_{x_2 = -a}^a \rd x_2 
\int_{y_1 = -b}^b \rd y_1 \int_{y_2 = -b}^b \rd y_2 \, \indicator{ (a-x_2)^2 + (y_1-y_2)^2 \ge 4 r^2 }\indicator{ -\epsilon/2<y_1< \epsilon/2}.
\end{equation}
This would mean that the first disk is hitting the hole. By interchanging the
indexes we would get an equal contribution from the other disk. There is a caveat: 
if the hole or the interaction radius is too large, we have to consider many
more cases for the calculation of the cross section area. The geometry of
the area for integration changes itself according to certain ranges.
We will consider then only small holes. We show a diagram 
in the figure \ref{diagintagu}.
\begin{figure}
\centering
\begin{tabular}{cc}
\includegraphics[width=0.4\textwidth]{diagramintegraagujero01.pdf} &
\includegraphics[width=0.4\textwidth]{diagramintegraagujero02.pdf}
\end{tabular}
\caption{Area of consideration for the integration of expression
 \ref{areaindicagujero}. The left side shows a hole small enough
so that the radius of interaction does not change the geometry of the
indicatrix set, the right side shows the other possibility.}\label{diagintagu}
\end{figure}
The integration is more tedious than in previous examples, as we cannot
take full advantage of the symmetries and we have to split it into
more cases, but still is elemental, and after a good day at the blackboard
it yields
\begin{align}\label{escape}
 A_{escape} &= 2 (w-2r) (h-2r) \epsilon.
\end{align}
\pmb{This is probably wrong.}


\subsection{Mean Sojourn  time for small disks and entrances}

We shall also test the validity of formula using the area in eq. \ref{escape}
for a small hole in the walls. If the formula is to be valid, we have
to make an adequate balance of the involved parameters. The hole has to
be only marginally larger than the disks, so that they have enough opportunity
to interact, thus achieving \emph{ergodic statistics}. A very large hole
on a wall with similar sized disks will result in only the disk closer to
it escaping. The average escape time then may be calculated as
a return time for the adequate section and we could use the following formula:
\begin{equation} 
\mean{\tau}_{sojourn} = 	
\frac{3 \pi}{4\sqrt{2}}
\frac {16a^2b^2-16\pi a b r^2 + \frac{64}{3} r^3 - 8 r^2  }
{4 a b \epsilon}.
\end{equation}
As can be expected, this time is proportional to the
free Volume available and inversely proportional to the size of the hole, $\epsilon$.
The formula should be adequately interpreted. This time would represent
the interval that the box, with the aforementioned hole, contains two disks,
starting to count at the moment that the second disks enters the box, and
stopping at the moment any of the disks leaves. 

