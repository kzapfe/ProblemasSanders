\documentclass[a4paper,10pt]{article}
\usepackage[utf8]{inputenc}

\usepackage{amsmath, amsfonts}

%opening
\title{Exact kinetics of two hard discs in a rectangular box}
\author{Rosa Rodríguez \& David P.~Sanders}
%\affiliation{Departamento de Física, Facultad de Ciencias, Universidad Nacional Autónoma de México, Ciudad Universitaria, Del.~Coyoacán, México D.F. 04510, Mexico}

\usepackage{mathptmx}

\newcommand{\defeq}{:=}
\newcommand{\mean}[1]{\left \langle #1 \right \rangle}

\newcommand{\rd}{\, \mathrm{d}}

\newcommand{\RR}{\mathbb{R}}

\newcommand{\vv}{\mathbf{v}}

% \newcommand{\indicator}[1]{\mathbf{1}_{ \left \{ #1 \right \} } }
\newcommand{\indicator}[1]{\mathbf{1}_{ \{   #1 \} } } 


\setlength{\parskip}{10pt}
\setlength{\parindent}{0pt}

\begin{document}

\maketitle
% 
 \begin{abstract}
%
We obtain exact results for the kinetics of the inertial motion of two hard discs in a rectangular two-dimensional box.
In particular, we calculate exactly the mean hopping time between exchanges of the horizontal positions of the discs, as well as mean collision rates between the two discs and between the discs and the box. We show numerically that the distributions are exponential?

 \end{abstract}

\section{Model}
We consider two discs of radius $r$ (and diameter $d=2r$) in a box of width $w$ and height $h$. The discs move inertially in the absence of forces, following straight-line trajectories, and undergo elastic collisions with each other and with the walls of the box.

We denote the position of the centre of the $i$th disc by $(x_{i}, y_{i})$ for $i=1,2$. Since the discs are hard, the disc centres are restricted to the region $(x_i, y_i) \in [-a,a] \times [-b, b]$, where $a \defeq a(d) \defeq \frac{w}{2} - r = \frac{w-d}{2}$ and $b \defeq b(d) \defeq \frac{h}{2} - r = \frac{h-d}{2}$.
%, and $x_i$ and $y_i$ are the Cartesian coordinates of the disc centres, with $i=1,2$.

The exclusion condition is $(x_1-x_2)^2 + (y_1-y_2)^2 \ge (2r)^2 = d^2$.
It is thus useful to work in terms of the new coordinates
\begin{equation}
 x \defeq \frac{x_1 - x_2}{\sqrt{2}}; 
\quad X \defeq \frac{x_1 + x_2}{\sqrt{2}}; 
\quad y \defeq \frac{y_1 - y_2}{\sqrt{2}}; 
\quad Y \defeq \frac{y_1 + y_2}{\sqrt{2}}.
\end{equation}


In these coordinates, we have the conditions $x \in [-a \sqrt{2}, +a \sqrt{2}]$ with $X \in [-a \sqrt{2} + |x|, a \sqrt{2} - |x|]$ and similarly $y \in [-b \sqrt{2}, b \sqrt{2}]$ and $Y \in [-b \sqrt{2} + |y|, +b \sqrt{2} - |y|]$,  with the constraint $x^2 + y^2 \ge \frac{d^2}{2}$.

%
%Better:
%\begin{equation}
% x \defeq \frac{x_1 - x_2}{2}; 
%\quad X \defeq {x_1 + x_2}; 
%\quad y \defeq \frac{y_1 - y_2}{2}; 
%\quad Y \defeq y_1 + y_2.
%\end{equation}

%Then $x \in [-a, +a ]$ with $X \in [-2a +2 |x|, +2a - 2|x|]$ and similarly $y \in [-b, +b]$ and $Y \in [-2b  + 2|y|, +2b  - 2|y|]$,  with the constraint $x^2 + y^2 \ge r^2$, where $r \defeq d/2$ is the disc radius.

Throughout, we suppose that we are in the regime for which the discs may pass vertically, i.e.\ such that $h > 2d$.

\textbf{What assumption do we use for the disc collision rate?}

MAYBE NEED ANOTHER SUPPOSITION FOR DISC COLLISION RATE?

\section{Mean free time}
A system of $N$ hard discs confined by hard walls may be treated as a billiard system REF, in which a single point  particle undergoes free motion between reflecting obstacles in a $2dN$-dimensional phase space, where the hard discs move in $d$-dimensional Euclidean space $\RR^d$, i.e.\ there are $d$ configuration-space variables in the original hard-disc system.

For such billiard systems, there is a very general exact result for the mean free time, i.e.\ the mean time between collisions of the particle with the walls REF Chernov. This can be thought of as a mean return time to the $(d-1)$-dimensional (i.e.\ codimension-$1$) cross-section given by the wall boundaries.
The general result is
\begin{equation}
 \mean{\tau} = \frac{|Q|}{|A|} \frac{|S^{d-1}|}{|B^{d-1}|}
\end{equation}
when the particle has velocity $1$.
Here, $|Q|$ denotes the $d$-dimensional volume of the available space in the billard and $|A|$ the $(d-1)$-dimensional volume of the cross-section.
 $|S^{d-1}|$ is the $(d-1)$-dimensional volume of the unit sphere in $\RR^d$, given by
\begin{equation}
  |S^{d-1}| = \frac{2 \pi^{d/2}}{\Gamma(d/2)},
\end{equation}
where $\Gamma(\cdot)$ is the gamma function. $|B^{d-1}|$ is the volume of the unit ball in $\RR^{d-1}$, given by $|B^{d-1}| = |S^{d-2}| / (d-1)$.

Machta and Zwanzig REF used a similar method to derive an escape time across a non-existent boundary by treating it as a recurrence time.
Since it is an escape time, they also used velocities whose components point only in a certain direction.

In our case, we are interested in the mean return time to a codimension-$1$ cross section, namely the set $\{x_1 = x_2\}$.
% This result is for a collision with a hard wall, so that only velocities from one side of the wall are allowed.
This surface $x_1=x_2$ is accessible from \emph{both} sides, so that there is an extra factor of two in the part of the velocities. Thus in our case we have
\begin{equation}
 \mean{\tau} = \frac{|Q| \, |S^{d-1}|} {2 \, |A| \, |B^{d-1}|}.	
\end{equation}

We must also take into account the mass $m$ of each particle and the total kinetic energy $E$.
We have in general $\sum_i \vv_i^2 = 2E / m$.
% This is for a system with energy $E=\frac{1}{2}$ and particles of mass $1$, so that $\sum_i \vv_i^2 = 1$, where $\vv_i$ is the velocity vector of the $i$th particle. 
% In general, we have $\sum_i \vv_i^2 = 2E / m$ for a system with total (kinetic) energy $E$ and all particles of mass $m$. In this case, each velocity is multiplied by
The above result, as derived by Chernov, was for the case $\sum_i \vv_i^2 = 1$, or $m=1$ and $E=\frac{1}{2}$.  For more general values of $E$ and $m$, we are simply working on a different energy surface in phase space. The particle trajectories are identical but their motion is a factor
$\sqrt{2E/m}$ faster, so that all times are reduced by this factor, giving the final result
\begin{equation}
  \mean{\tau} = \frac{1}{\sqrt{2E / m}} \frac{|Q| \, |S^{d-1}|} {2 \, |A| \, |B^{d-1}|}.	
\end{equation}


\section{Calculation of volumes for the two-disc problem}
We return to the problem of two hard discs in a box. From the previous section, we must calculate the volume $|Q|$ of the available space, and the volume $|A|$ of the cross-section $\{x_1 = x_2\}$.

\subsection{Volume of available space}
% Both disc centres live in $(x_i, y_i) \in [-a, a] \times [-b, b]$, and we also have the restriction $(x_1
\begin{align}
 V = \int_{x_1 = -a}^a \rd x_1 \int_{x_2 = -a}^a \rd x_2 
\int_{y_1 = -b}^b \rd y_1 \int_{y_2 = -b}^b \rd y_2 \, \indicator{ (x_1-x_2)^2 + (y_1-y_2)^2 \ge d^2 },
\end{align}
where $\indicator{Z}$ indicates the indicator function of the set $Z$, given by $\mathbf{1}_Z (x) = 1$ if $x \in Z$, and $=0$ if $x \notin Z$, which restricts the integral to the desired region $Z$.

Since the region $Z$ of interest involves relative positions, we change variables via an orthogonal transformation (i.e., a coordinate rotation, which thus has Jacobian $1$), giving
\begin{equation}
 V = \int_{x=-a \sqrt{2}}^{a \sqrt{2}} \rd x 
\int_{X=-a \sqrt{2} + |x| }^{a \sqrt{2} - |x|}  \rd X
 \int_{y=-b \sqrt{2}}^{b \sqrt{2}} \rd y
\int_{Y=-b \sqrt{2} + |y| }^{b \sqrt{2}-|y|}  \rd Y
\, \indicator{ x^2 + y^2 \ge \frac{d^2}{2}  }.
\end{equation}
In these coordinates, $X$ and $Y$ do not appear in the integrand, so that these integrals may be done trivially, thus reducing to the following two-dimensional integral:
\begin{align}
 V &= \int_{x=-a \sqrt{2}}^{a \sqrt{2}} \rd x  \int_{y=-b \sqrt{2}}^{b \sqrt{2}} \rd y
\, 2 \left( a \sqrt{2} - |x| \right) \, 2 \left( b \sqrt{2} - |y| \right) \,  \indicator{ x^2 + y^2 \ge \frac{d^2}{2} } \\
&= 16 \int_{x=0}^{a \sqrt{2}} \rd x  \int_{y=0}^{b \sqrt{2}} \rd y
\, \left( a \sqrt{2} - x \right) \, \left( b \sqrt{2} - y \right) \,  \indicator{ x^2 + y^2 \ge \frac{d^2}{2} },
\end{align}
using the symmetry.

Thus $V = 16(I_1 + I_2)$, where $I_1$ and $I_2$ are the resulting integrals over the regions where the available region for $y$ is affected by, and is not affected by, respectively, the restriction to lie outside the disc.
We have
\begin{align}
 I_1 &= \int_{x=0}^{d / \sqrt{2}} \left[ \int_{y = \sqrt{\frac{1}{2} {d^2} - x^2}}^{b \sqrt{2}} \left( b \sqrt{2} - y \right) \rd y \right]  \left( a \sqrt{2} - x \right) \rd x \\
&= 	
a b^{2} d + \textstyle \frac{1}{6} (a+b) d^{3} - \frac{1}{32}  d^{4} - \frac{1}{4} {\left(\pi a b + b^{2}\right)} d^{2},
\end{align}
and
\begin{align}
 I_2 &= \int_{x=d / \sqrt{2}}^{a \sqrt{2}} \left[ \int_{y = 0}^{b \sqrt{2}} \left( b \sqrt{2} - y \right) \rd y \right]  \left( a \sqrt{2} - x \right) \rd x \\
&=	
{\left( a^{2} - a d + \textstyle \frac{1}{4}  d^{2}\right)} b^{2}.
\end{align}
Thus 
\begin{equation}
 V %= 16(I_1 + I_2) =     	
= 16 a^{2} b^{2}  - 4 \pi a b d^{2} + \textstyle \frac{8}{3} (a+b) d^{3}  - \frac{1}{2} d^{4},
\end{equation}
as was previously obtained by Munakata and Hu 2002.

Note the volume available to place two discs inside the hard box but \emph{without} the 
 hard-disc exclusion constraint is simply given by the first term, $16 a^2 b^2$; thus, the remaining terms represent the excluded volume due to this constraint.

We have checked this result with simple Monte Carlo simulations, by generating random positions for the sphere centres in $[-a,a] \times [-b,b]$ uniformly and 
counting the proportion of such initial conditions for which the two spheres do not overlap.


%  $x$, and does not depend on $x$, respectively



\subsection{Volume of cross-section $\{x_1 = x_2\}$}
The calculation of the $3$-dimensional volume of the cross-section $\{x_1 = x_2\}$, which becomes $\{ x=0 \}$ in the new coordinates, proceeds in a similar way:
[It seems to me that this just consists of inserting $\delta(x-0)$ in the integrand of the 4D integral?]
\begin{align}
 A &= \int_{X=-a \sqrt{2} }^{a \sqrt{2}}  \rd X
 \int_{y=-b \sqrt{2}}^{b \sqrt{2}} \rd y
\int_{Y=-b \sqrt{2} + |y| }^{b \sqrt{2}-|y|}  \rd Y
\, \indicator{y^2 \ge \frac{d^2}{2} } \\
&= 8 a \sqrt{2} \int_{y=0}^{b \sqrt{2}} \left( b \sqrt{2} - y \right)  \indicator{y \ge \frac{d}{\sqrt{2}}}  \rd y \\
&= 8 a \sqrt{2} \int_{y= \frac{d}{\sqrt{2}}}^{b \sqrt{2}}  \left( b \sqrt{2} - y \right)  \rd y \\
&= 2 \sqrt{2} a ( 2b - d )^2.
% b^{2} -  b d + \frac{d^{2}}{4} \right) .
\end{align}

We have checked this result with simple Monte Carlo simulations, by counting the proportion of successful placements of hard discs for which the distance $|x_1 - x_2|$ is within a small tolerance of $0$.


\subsection{Mean hopping time}
The system of two hard discs is equivalent to a billiard in $d=4$ dimensions, so that
\begin{equation}
 |S^{d-1}| = |S^3| = 2 \pi^2; \qquad |B^{d-1}| = 4 \pi / 3.
\end{equation}

Inserting these results into the formula for the mean hopping time gives
\begin{equation}
 \mean{\tau} = 	
% \frac{ \pi \left( 16  a^{2} b^{2} + (a+b) d^{3}   - 3  \pi a b d^{2} - \textstyle \frac{3}{16}  d^{4} \right) }
% { 2 {\left( 4  \sqrt{2} a b^{2} - 4  \sqrt{2} a b d + \sqrt{2} a d^{2}\right )}}
\frac{3 \pi}{4}
\frac
{16 a^{2} b^{2}  - 4 \pi a b d^{2} + \textstyle \frac{8}{3} (a+b) d^{3}  - \frac{1}{2} d^{4}}
{ 2 \sqrt{2} a ( 2b - d )^2}.
\end{equation}

Substituting the expressions for $a(d)$ and $b(d)$ in terms of $w$, $h$ and $d$ gives
the main result of this paper, the exact expression for the mean hopping time:
\begin{equation}
 \mean{\tau} = 
\frac
{ \pi {\left[ 6\,{\left(d-w\right)}{\left(d-h\right)}\pi d^{2}-6 \, 
{\left(d-w\right)}^{2}{\left(d-h\right)}^{2}
+ 8 \, {\left(d-w\right)} d^{3}+8\,{\left(d-h\right)}d^{3}+3\,d^{4}\right]}
  }
{ 	
8 \sqrt{2} \,{\left(d-w\right)}{\left(2\,d-h\right)}^{2}  }.
\end{equation}

We expect this to blow up when the discs can almost not pass each other, i.e.\ when $d \to h/2$, so we 
% Due to the form of the denominator, it looks like this blows up as $\epsilon^{-2}$, as predicted by heuristic arguments.
substite $d =: h/2 - \epsilon$, where  $2 \epsilon$ is the available vertical space for the two discs to pass each other vertically,
to obtain the asymptotic behaviour when $\epsilon \to 0$:
\begin{equation}
\tau \sim \frac{4\,{\left(2\,\pi+3\,\pi^{2}\right)}h^{3}w - 3\,{\left(\pi+2\,\pi^{2}\right)}h^{4}-24\,\pi h^{2}w^{2}}
{256 \sqrt{2} \,{\left(h - 2 w \right)}}  \epsilon^{-2}.
\end{equation}


\section{Lo que falta}

Comparison with numerics: is perfect!

To do: comparison of our mean hopping time with the mean *first* time to hop, which is what is studied in other papers.
Hopefully (and presumably) they are the same asymptotically when $\epsilon \to 0$?

For square box, can escape in either direction, so hopping time (including in other direction) should be half?

3D?  Must be long channel. Spheres may be confined in 1 direction  and only able to exchange in other direction, or be able to exchange in two directions.
This probably affects asymptotics.

Escape time through a hole in the wall?? (Probably leave for future.)

Mean time between disc collisions, collisions with walls. Should be easy with the formula.

Distributions of all these times (numerical with histogram).




\end{document}
