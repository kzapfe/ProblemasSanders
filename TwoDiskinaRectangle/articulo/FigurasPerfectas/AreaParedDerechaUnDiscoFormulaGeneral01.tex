\documentclass[superscriptaddress,preprint,showpacs, single-column]{revtex4-1}

\usepackage[utf8]{inputenc}
\usepackage{amsmath, amsfonts}
\usepackage{graphicx}
\usepackage{natbib}
\usepackage{caption}
\usepackage{subcaption}
%\usepackage{authblk} parece ser que ya esta metido cuando usas preprint

\usepackage{hyperref}



\usepackage{mathptmx}
% I dont know if that package is compatible with revtex.

\newcommand{\defeq}{:=}
\newcommand{\mean}[1]{\left \langle #1 \right \rangle}
\newcommand{\rd}[1]{\mathrm{d}{#1} \,}
\newcommand{\RR}{\mathbb{R}}
\newcommand{\vv}{\mathbf{v}}
\newcommand{\indicator}[1]{\mathbf{1}_{ \{   #1 \} } } 
\newcommand{\etal}{et al.\ } 

\setlength{\parskip}{10pt}
\setlength{\parindent}{0pt}



\begin{document}

\title{Exact and General Expression for the Area.}

First, reminder of the usual definitions:\\
$w:=$ width of the table. Paralel to the axis for $x_1,x_2$ \\
$h:=$ height of the table. Paralel to the axis for $y_1,y_2$ 

We denote the position of the center of the $i$th disc by 
$(x_{i}, y_{i})$ for $i=1,2$. Since the discs are hard, 
the disc centers are restricted to the region 
$(x_i, y_i) \in [-a,a] \times [-b, b]$, where 
$a \defeq a(r) \defeq \frac{w}{2} - r $ and
$b \defeq b(r) \defeq \frac{h}{2} - r $.

The Area that represents collitions with a wall would be the
integral of the configuration space outside the exclusion condition
and a Dirac Delta that represents $x_1=a(r)$.

We shall use the following auxiliary functions:

\begin{align}
  \alpha(r)=\arcsin(b(r)/r) \\
  \beta(r)=\arccos(b(r)/r) \\
  \cos \alpha := \sqrt{r^2-b^2}/r \\
  \sin \beta := \sqrt{r^2-a^2}/r
\end{align}



It is convenient to separate the expression for the area into
``positive'' and ``negative'' areas
\begin{equation}
A_{pd1}(r)=A_{pos}(r)-A_{neg}(r)
\end{equation}

The positive contribution is nicely wrapped in the expression:
\begin{equation}
  A_{pos}(r)=8ab^2
\end{equation}
The other one is not so direct:
After a fun and enjoyable derivation I ended with:
\begin{equation}
  A_{neg}=\begin{cases}
  8(r^2b \pi/2 -2r^3/3) \text{ if } r \le h/4 \\
  8(r^2b \alpha +2r^3/3 (\cos\alfa -1))+8/3b^2\sqrt{r^2-b^2}
  \text{ if }  h/4 < r \leq w/4 \\
  8(r^2b (\alpha-\beta) +2r^3/3 (\cos\alfa -a/r))+8/3b^2\sqrt{r^2-b^2}
  +8(ab\sqrt{r^2-a^2}-1/3 a(r^2-a^2))
  \text{ if }  h/4 < r \leq w/4 \\
  \end{cases}
\end{equation}


\end{document}


