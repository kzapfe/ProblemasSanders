\documentclass[letterpaper,12pt]{article}

\usepackage[utf8]{inputenc}
\usepackage{amsmath,amssymb}
\usepackage{graphicx}
\usepackage{caption}
\usepackage{subcaption}
%Nueva convenvion para las subfigures es el subcaption package

\usepackage[spanish]{babel}
\usepackage{bm}
\usepackage[cm]{fullpage}

\bibliographystyle{alpha}


\newcommand{\xfase}{\mathbf{x}}
\newcommand{\qfase}{\mathbf{q}}
\newcommand{\pfase}{\mathbf{p}}
\newcommand{\xifase}{ {\boldsymbol{\xi}} }
\newcommand{\mufase}{ {\boldsymbol{\mu}} }
\newcommand{\Ifase}{\mathbf{I}}
\newcommand{\Pfase}{\mathbf{P}}
\newcommand{\Scat}{\mathbf{S}}
\newcommand{\Jsimp}{\mathbf{J}}
\newcommand{\Dom}{\mathbb{D}}
\newcommand{\Var}{\mathbb{M}}
\newcommand{\bra}[1]{\langle #1|}
\newcommand{\ket}[1]{|#1\rangle}
\newcommand{\braket}[2]{\langle #1|#2\rangle}



\DeclareMathOperator*{\cod}{cod}
\DeclareMathOperator*{\traza}{traza}


\title{Sobre la longitud y distribución de las trayectorias balísticas y cuasi balísticas en una latiz de billares Cilíndricos.}
\author{ W. P. Karel Zapfe\\Fac. de Ciencias, UNAM}


\begin{document}

\maketitle

\abstract{
Aquí analizamos la distribución de órbitas que se propagan de forma balística
en una latiz cuadrada o cúbica de billares cilíndricos. Demostramos como
las transformaciones lineales del cilindro en sí mismo son la dinámica fundamental
a resolver y que de las recurrencias de los mapas del círculo en si mismo se
encuentra la causa del quiebre de las orbitas periódicas.
}

\section{Introducción}

Los billares cilíndricos truncados con cortes a $\pi/4$ respecto al eje 
resultan tener una dinámica bastante rica. La motivación inicial
fue obtenida por una representación de la dinámica de dos partículas
encerradas en una caja atadas por un hilo \cite{SandersGilbert}. 
Si la caja sólo tiene fronteras paralelas e infinitas en una dirección,
una transformación lineal de coordenadas produce
un sistema interpretable como  billar
en el interior de una cuña cilíndrica (ver figura \ref{cunha01}).
 Este problema produce una dinámica bastante complicada, y tiene una
sensibilidad curiosa dependiendo de su aspecto geométrico que varía entre 
varios régimenes de estabilidad según Liapunov.
Podemos hacer el desdoble
especular sobre la frontera del billar, primeramente
a través de las superficies semi elípticas de la cuña,
obteniendo así una cruz.
El desdoble subsecuente sobre las
caras circulares de la cruz produce una latiz
de cilindros ortogonales como se muestra
en la figura \ref{latiz01}. El billar desdoblado
puede modelar transporte de partículas en sistemas porosos,
tales como areogeles de distintas densidades \cite{Veamos}.
Querríamos saber si el transporte en dicho sistema es normalmente
difusivo o presenta cualidades diferentes. Para ello
tendríamos que encontrar si la latiz
presenta preferencia por mantener el transporte en una
dirección establecida, o, a \emph{grosso modo} se comporta como
un caminante aleatorio equiprobable en una red bidimensional.
Para ver que tan válida es esa aproximación, tenemos que ver
que conjunto de trayectorias se comporta
efectivamente como si tuviera \emph{horizonte infinito}, es
decir, que no haya nada en su camino que les haga cambiar de dirección
de propagación, y, si no hay conjunto de medida significativa de ellos,
como es que el número de orbitas con \emph{horizontes muy lejanos}
va disminuyendo con el tiempo o distancia \cite{SandersHorizon}. 


\begin{figure}[ht]
  \centering
  \begin{subfigure}[c]{0.5\textwidth}
    \centering
          \includegraphics[width=\textwidth]{DosParticulasAmarradasCanalInfinito01.pdf}
                \caption{Sistema original}
                \label{particulasamarradas}
  \end{subfigure}%
\hfill
\begin{subfigure}[c]{0.4\textwidth}
    \centering
          \includegraphics[width=\textwidth]{CunhaCilindricaTrayectoriasTransparenteChingona02.png}
          \caption{Sistema equivalente}      
          \label{cunhadeparticlas}
  \end{subfigure}%
\caption{Una interacción atractiva dura en un canal infinito obtenida por un ``hilo'' infinitamente
rígido. El sistema es equivalente a un billar en una cuña cilíndrica.}\label{cunha01}
\end{figure}


\begin{figure}[ht]
  \centering
  \begin{subfigure}[c]{0.5\textwidth}
    \centering
          \includegraphics[width=\textwidth]{CruzyTrayectorias02.png}
                \caption{Desdoble a través de las caras semi-elípticas.}
                \label{Cruz01}
  \end{subfigure}%
\hfill
\begin{subfigure}[c]{0.4\textwidth}
    \centering
          \includegraphics[width=\textwidth]{CruzLatizyTray01.png}
          \caption{Desdoble continuado}      
          \label{CruzLatiz}
  \end{subfigure}%
\caption{Al desdoblar el sistema de la cuña, obtenemos una latiz infinita de cilindros
cruzados.}\label{latiz01}
\end{figure}



\section{El Modelo}

\subsection{Difusión en una latiz de cruces.}

Imaginemos un punto arbitrario dentro de uno de los cilindros
que conforman la latiz. De ese punto tomemos el haz de
todas las direcciones posibles, que forma una esfera.
Cada rayo con origen en ese punto podría chocar con la
frontera definida por el cilindro con el que estamos, 
o con cualquiera de los que lo atraviesan ortogonalmente.
Mientras lo haga con el primer cilindro, su momento
angular y momento axial se mantendrán constantes,
por lo tanto, su difusión será balística, hasta que
se estrelle con algún otro cilindro. Se estrellaría con aquellos
segundos cilindros si pasara a través de la superficie del hueco
de la intersección de los segundos con el primero, así que
basta considerar como terminada una trayectoria balística
que choque contra alguna de estas áreas especiales.

\begin{figure}
  \centering
  \includegraphics[width=0.45\textwidth]{CilindroConAreaProhibida01.png}
  \caption{El área de los hoyos en una sección periódica del cilindro.
    Repitiendo esta figura a lo largo del eje del cilindro obtenemos
    todos los hoyos que indican un cambio del momento axial y angular.}
  \label{huecos01}
\end{figure}


Consideremos primero un billar cilíndrico infinito con ciertos
sectores de su superficie marcadas (véase la fig. \ref{huecos01}). 
El área marcada corresponde
a la intersección con los billares transversales. Dada la geometría
de la intersección (el cilindro transversal se encuentra por fuera del 
área virtual del otro), un golpe en la zona marcada es equivalente
a una partícula que cambió su dirección de propagación. 
Llamaremos trayectorias balísticas no sólo aquellas que viajan sin golpear nada,
sino a todas ellas que viajan sin golpear las secciones marcadas, que de
ahora en adelante llamaremos hoyos. En este primer caso, los hoyos están
delimitados por arcos de elipse, para ser exactos cada arco es media
elipse cortada por el semieje menor. 

Escojamos un punto arbitrario dentro del cilindro.
Sin pérdida de generalidad, podemos decir que su altura, que identificaremos
como la coordenada $z$ de un sistema cartesiano, es cero, y 
que se encuentra sobre el eje $x$, digamos pues que sus coordenadas
son $\qfase=(q_x,0,0)$. Este punto conlleva la esfera $S^2$ de todas las direcciones
posibles. Escodamos unidades tales que el cilindro tenga radio $1$ y
que la magnitud de la velocidad sea también $1$. 
Entonces el problema es equivalente a proyectar una esfera desde su centro
 sobre la superficie del cilindro. Nótese que lo único que necesitamos
es que la esfera tenga el centro dentro del cilindro, el tamaño es irrelevante,
ya que su medida está determinada por la magnitud de la velocidad. Su
área es, en nuestras unidades, siempre igual a $4\pi$ y está uniformemente
distribuida. Nuestro primer resultado será la distribución de esta densidad
de medida sobre el cilindro. Lo que nos interesa en primera instancia
es la medida sobre la esfera de la preimagen del área de los hoyos.

Un par de diagramas (en la figura \ref{diagramaschoque} )
y una simple aplicación de trigonometría nos dará
las coordenadas de cada punto de choque a partir del punto inicial.
El choque está dado por la cantidad de momento distribuido en
las direcciones $x,y$. Usando las variables coordenadas
esféricas usuales, 
$v=(\cos \phi \cos\theta, \sin \phi \cos\theta, \sin \theta)$,
 la cantidad $\|v_{plano}\|= \cos \theta$ es la rapidez 
con la cual encontraremos la frontera. La distancia se puede determinar
de la figura \ref{Diag01} y una simple aplicación
de la ley de los cosenos. Dado que el coseno del ángulo interior
y exterior de un par de rayos coinciden, podemos afirmar que la 
distancia a la pared está determinada por

\begin{figure}[ht]
  \centering
  \begin{subfigure}[c]{0.4\textwidth}
    \centering
          \includegraphics[width=\textwidth]{Diagramachoque01.pdf}
                \caption{Corte Transversal}
                \label{Diag01}
  \end{subfigure}%
\hfill
\begin{subfigure}[c]{0.4\textwidth}
    \centering
          \includegraphics[width=\textwidth]{Diagramachoque02.pdf}
          \caption{Corte Axial}      
          \label{Diag02}
  \end{subfigure}%
\caption{Diagramas para la proyección de una esfera en el cilindro.}
\label{diagramaschoque}
\end{figure}

\begin{equation}
q_x^2+d^2-2q_x d \cos\phi=1.
\end{equation}

Sólo nos interesa la solución mayor de esta expresión cuadrática,
así que la distancia en cuestión es:

\begin{equation}\label{distanciaplano}
d=q_x \cos\phi+\sqrt{ 1- q_x^2 \sin^2 \phi } 
\end{equation}

La función de $d(q_x, \phi)$ se puede ver en la figura \ref{dgrafica}.
Podemos apreciar que si $q_x=-1$, entonces todo el rango de
direcciones que apuntan a la pared más cercana dan $d=0$, como debe ser. 
Por simetría, sólo presentamos valores entre $-1$ y $0$, correspondiente
este último a la proyección de Mercator. 

\begin{figure}[ht]
  \centering
  \begin{subfigure}[c]{0.47\textwidth}
    \centering
    \includegraphics[width=0.95\textwidth]{D-en-funcion-q-phi.pdf}
    \caption{ $d(q_x, \phi)$}\label{dgrafica}
  \end{subfigure}
  \hfill
  \begin{subfigure}[c]{0.47\textwidth}
    \centering
    \includegraphics[width=0.95\textwidth]{z-en-funcion-theta-pi-q-050.pdf}
    \caption{ $z(\phi, \theta), q_x=-0.5$}\label{dgrafica}
  \end{subfigure}
  \caption{Las distancias relevantes del mapeo de la esfera en el cilindro,
$d,z$.}\label{graficasmapeo01}
\end{figure}
    


De la expresión obtenemos el tiempo de choque, $t=d/\|v_{plano}\|$ y de ahí
la altura del punto imagen, $z=d \tan \theta$. Con esto, fijo el valor
$q_x$ tenemos entonces la proyección de la esfera de velocidades sobre
el cilindro:

\begin{equation}\label{SuperMercator}
T_{q_x}(\phi,\theta)=( d \cos \phi +q_x, d \sin \phi, d \tan\theta)
\end{equation}

Esta última expresión resulta práctica para obtener la imagen de cada
punto de la esfera de velocidades con origen en un punto $\qfase_c$ del
espacio de configuración. La generalización es trivial, consistente
en una simple rotación de los ejes $x,y$ y una traslación a lo
largo del eje de coordenadas. Sin embargo resulta particularmente
incómoda para calcular algo que nos interesa más, que es la imagen
inversa de los agujeros en una esfera de velocidades con pie en un
punto arbitrario dentro del cilindro. 
Es claro que lo que estamos haciendo simplemente es obtener una proyección
del cilindro en una esfera con centro arbitrario $\qfase_c$. La proyección
cartesiana
de cualquier punto del espacio en esta esfera es dividir sus coordenadas entre
la distancia que separa al punto del cento: 

\begin{equation}
P(x,y,z)=\frac{(x,y,z)}{\|(x-q_x,y-q_y,z-q_z)\|} \in S^2 
\end{equation} 

Con esto ya podemos visualizar el conjunto de vectores velocidad
que escapan \emph{antes del primer choque} del cilindro, es decir, que
se encuentran con uno de los hoyos. Un par de ejemplos se puede ver en la figura
\ref{esferaconhuecos}. Recordemos que el parámetro $h$ controla la distancia
entre intersecciones del cilindro, siendo está, medida de centro a centro,
$2h+2$. En el caso de que $h=0$ (el cílindro se reduce a la cuña), los huecos
son tangentes entre ellos y por lo tanto sus imágenes en la esfera también. Esto
produce el curioso hecho de que la esfera se encuentra dividida por proyecciones
de dos elementos romboidales. La longitud balística de las trayectorias que no
escapan al primer rebote está simplemente determinada por el valor $\cos \theta$
en las coordenadas de la esfera. 

\begin{figure}
  \centering
  \includegraphics[width=0.75\textwidth]{EsferaConHuecosCilindroSimple01.png}
  \caption{Un caso particular de la proyección de los hoyos en la esfera,
    los parámetros son $\qfase=(-0.5,0,0), h=0.5$. La resolución de la imagen permite
ver hasta 5 niveles de hoyos en el cilindro, hacia arriba y abajo.}
  \label{esferaconhuecos}
\end{figure}

Un poco más dificil es conseguir ver cuales son las trayectorias que escapan
después del primer choque con el cilindro. Sin embargo esto no 
pasa de ser un ejercicio algo fastidioso de trigonometría, dado que una vez
dado el vector velocidad y el punto de origen $\qfase$, el momento angular
y la altura que separa cada choque están determinados. Así que resolviendo
el segundo choque, resolvemos todos. Hay una serie de consideraciones previas.

Las condiciones iniciales que dan momento angular axial constante
dibujan, sobre la superficie de la esfera, círculos. Usando las coordenadadas de
la figura \ref{diagramaschoque}, las curvas de momento angular axial
constante están dadas por 

\begin{equation}
 \cos \theta \sin \phi = L_z /q_x.
\end{equation}

Es fácil ver que el máximo momento angular se alcanza en la
dirección $\theta=0, \phi=3 \pi /2$ y el mínimo en la opuesta,
siendo el meridiano $\phi=0,\pi$ la linea nodal sobre la esfera. Los
vectores velocidad que forman un cono con eje en $\phi= \pi /2, 3\pi /2$ 
producen todos el mismo valor para $L_z$

Cada uno de los puntos imágen de estos círculos rotará alrededor
del cilindro un ángulo constante con cada choque, mientras se
desplaza en la dirección vertical de forma equiespaciada.
El ángulo en cuestión es 
\begin{equation}
\psi=2 \arccos (q_x \sin \phi)
\end{equation}

El tiempo que le toma a la partícula llevar a cabo este choque
es la longitud de cuerda dividia entre la rapidez planar  $\cos \theta$.
De ahí podemos obtener el cambio en la altura con cada choque:

\begin{equation}
  z\rightarrow z+2\tan(\theta)\sqrt{1-q_x^2\sin^2\phi}
\end{equation}

Mientras una partícula no choque contra un hoyo, se propagará
balísticamente en la dirección $z$ al ritmo arriba mostrado.
Ahora bien: propongo proyectar cada imágen producida de este mapeo
sobre la esfera otra vez. La razón es que en la esfera tenemos la 
información del momento angular y la velocidad balística $v_z$

Un problema de considerar la iteración de los choques tanto
sobre la esfera como sobre el cilindro es que no inducen un mapeo
uno a uno. De hecho ninguna medida que no esté concentrada
en los polos de la esfera se conserva con este mapeo, lo cual 
lo hace terriblemente inconveniente desde el punto de vista
de teoría ergódica. Así que tendríamos que remitir todas las
medidas a la esfera de condiciones iniciales. 

\subsection{Una version plana del problema}

Uno podría pensar que resolviendo el problema por secciones 
a lo largo de $\phi= cte$ pero no adelanta en nada eso. Dado
que excepto en $\phi=0$ tenemos un conjunto donde todos
tienen el mismo momento angular, es el único con el que podemos
hacer un estudio descente. Estudiaremos ese caso
para ganar un poco de percepción. Supongamos que tomamos las condiciones
iniciales con velocidad $\phi=0,pi$, es decir, $L_z=0$, que se hayan
alrededor de la esfera en un círculo máximo que cruza los polos. 
De forma más general, dicho círculo no debe estar alineado
en ninguna forma con el plano  que contiene a los cilindros 
que intersecan a nuestro cilindro, es decir, no está alineado en 
forma alguna con los agujeros. Digamos que hay una diferencia
de $\psi$ entre estos dos planos, siendo los hoyos centrados
en $\psi$ (ver diagrama en la figura \ref{diagramaforplano01}).

 
\begin{figure}[ht]
\centering
\begin{subfigure}[c]{0.6\textwidth}
  \centering
  \includegraphics[width=0.45\textwidth]{Diagramachoque03.pdf}
  \caption{Corte transversal.}
\end{subfigure}
\begin{subfigure}[c]{0.32\textwidth}
  \centering
  \includegraphics[width=0.85\textwidth]{Diagramachoque04.pdf}
  \caption{Corte coaxial.}
\end{subfigure}
\caption{Resulta comodo usar las coordenadas ya fijas
para la esfera de las velocidades, pero considererar que
los huecos no están alineados con el eje $x$. Aquí la
zona roja indica la posible posición de los huecos en un
corte transversal y el angulo $\psi$ su alineación central. }
\label{diagramaforplano01}
\end{figure}

La partícula que comience en éste circulo máximo se mantendrá
en un plano que contiene al eje del cilindro. 
Desde el punto de vista de la partícula, ella se encontrará
rebotando entre dos paredes rectas con hoyos de tamaño
$2\cos\psi$ centrados a la misma altura como
en la figura \ref{diagramaforplano02}.

De esos diagramas es claro que todas las trayectorias excepto un conjunto 
racional de medida zero terminan escapando por los huecos. 
Simplemente podemos desdoblar la trayectoria a lo largo de las
paredes y marcar los huecos como obstaculos.
De ahí se puede ver que el conjunto de trayectorias que pasan
entre todos los obstaculos son casos aislados. Para formalizar 
ese efecto podemos hacer lo siguiente: una vez desdoblada
la órbita a travez de las paredes rígidas, podemos considedarar
los hoyos como una latiz de paredes absorventes. Nuestro único
interés es ver cuanto se propagan en la direcció n $z$ positiva, asi
que integramos el ángulo que proyectan todos los absorventes
hasta cierta distancia. Dado que forman una latiz, resulta conveniente
hacerlo por pasos iguales a la longitud de la celda. Podemos
apreciar el proceso en la figura \ref{RedAbsorvers}.
Para ser consistenes, supongamos que el cero del eje $z$ coincide con 
el centro de la longitud de los absorvedores. Los parámetros
iniciales son entonces $(q_x, q_z)$ para el pie de las velocidades
iniciales. El primero por simetría lo consideraré $\in (-1, 0]$,
y el segundo tiene que estar $\in (0, 2h+2)$.

Vamos a llamare $\alpha_kj$ a los angulos subtendidos por
 aquellos absorvente que se encuentran
a la derecha en el diagrama, y $\beta_kj$ a los de la izquierda. 
Si la partícula se encuentra entre las alturas $(-\cos \psi, \cos \psi)$,
los absorventes más cercanos estarán asociados a los ángulos
$\alpha_0, \beta_0$. Dado que estos cubren completamente la posibilidad
de topar uno detrás de ellos, sólo es necesario un índice 
para localizarlos. El índice $j$ indicará la altura, el índice
$k$ la posición horizontal (podríamos utilizar $k<0 $ para evitar
la $\beta$).  

Las relaciones que describen el choque con un absorvente son

\begin{align}
q_x+t \cos \theta & = k, \\
q_z+t\sin \theta  & \in  (j (2h+2)-\cos \psi, j (2h+2)+\cos \psi ).
\end{align}

Buscamos siempre la solución con el valor de $k$ más pequeño. 


\end{document}


