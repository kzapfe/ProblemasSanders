\documentclass[a4paper,12pt]{article}

\usepackage[utf8]{inputenc}
\usepackage{amsmath,amssymb}
\usepackage{graphicx}
%\usepackage{subfigure}
%\usepackage[spanish]{babel}
\usepackage{bm}
\usepackage{caption}
\usepackage{subcaption}

\usepackage[cm]{fullpage}
\usepackage[light]{antpolt}
\usepackage[T1]{fontenc}

\usepackage{float}

% al final de cada seccion
\usepackage[section]{placeins}


\bibliographystyle{alpha}


\title{Allmost Ballistic Behaviour on Piecewise Cylinder Billiards}
\author{W. P. K. Zapfe}


\begin{document}

\maketitle
\abstract{
The orthogonal network of cyllindrical conductors with billiard
dynamics appears to show diffusive behaviour, but with some unexpected
features. In order to make a claim of standard diffusive behaviour,
we must reduce the influence of ballistic trayectories to a negligible
set.}



\section{The trayectories}

The motion inside the 
cyllinder is completelly integrable,
so encounter times for specific parts of the wall are
highly dependant on the initial conditions, which essentialy means that an 
ergodic approximation would be unapplicable. 


\section{Ballistic trayectories inside a cyllinder}

In this work I refeer to ballistic trayectories any trayectories
which do not hit an \emph{exit} from the billiard. The
particles which bounce within the walls of the cyllinder are considered
ballistics, as the value of the momentum parallel to the axis is
conserved. I designate this by $p_z$. In other systems ballistic
trayectories can spread in a fan of trayectories, here
they are constrained to avance allong this $z$ axis. Notice that
a particle can hit the walls of the transversal cyllinder and still
continue inside the first one. I do not call this a ballistic
trayectory because its axial momentum changes in such a bounce. 
Still, we may have to take into account the intuitivelly unlikely
case that a trayectorie who undergoes this kind of bounces interchanges
energy between its degrees of freedom in a way wich makes it \emph{gain}
$p_z$ momentum. For reasons that shall become aparent in the
next paragraphs, those trayectories would count as ``exit and reentrance'' 
events. 



 



\end{document}
