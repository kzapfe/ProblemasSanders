\documentclass[letterpaperr,12pt]{article}

\usepackage[utf8]{inputenc}
\usepackage{amsmath,amssymb}
%\usepackage{commath}
\usepackage[T1]{fontenc}
%\usepackage{garamond}
%\usepackage[garamond,cmbraces]{newtxmath}
\usepackage{natbib}

\usepackage{xspace}
\usepackage{graphicx}

\usepackage{float}
\usepackage{caption}
\usepackage{subcaption}
%\usepackage{anysize}
\usepackage[margin=2cm]{geometry}

%\usepackage{bm}

\bibliographystyle{alpha}

\newcommand{\Acase}{\textbf{A}\xspace}
\newcommand{\Bcase}{\textbf{B}\xspace}
\newcommand{\Ccase}{\textbf{C}\xspace}
\newcommand{\Dcase}{\textbf{D}\xspace}
\newcommand{\Lagr}{\mathcal{L}\xspace}
\newcommand{\Uno}{\textbf{1}\xspace}
\newcommand{\Dos}{\textbf{2}\xspace}

\title{Two disks in a circle}
\author{W. P. K. Zapfe}

\begin{document}

\section{Some Lagrangian considerations}

The system consists on two disks following usual billiard dynamics 
\cite{markchern} inside
a table delimited by a circular wall. The disks have no internal degrees
of freedom, so we can describe their dynamics using only the coordinates 
of their centre.
The billiard border has ratio of one arbitrary unit 
and both disks inside have ratio
$r$ and  mass equal to one arbitrary unit. We will call the discs as
\Uno and \Dos, and give subscripts to the symbols denoting
their properties
accordingly.

The available  configurations space is determined by the billiard
table and the collision condition:

\begin{equation}
x_i^2+y_i^2 \leq (1-r)^2 \text{ for } i=1,2,
\end{equation}

\begin{equation}
(x_1-x_2)^2+(y_1-y_2)^2 \geq (2r)^2.
\end{equation}
 
The energy is conserved, being equal to the kinetic energy:

\begin{equation}
E=\frac{m_1(\dot{x_1}^2+\dot{y_1}^2)+m_2(\dot{x_2}^2+\dot{y_2}^2)}{2}.
\end{equation}

By Noether's Theorem we also known that the total angular momentum is conserved.
We will denote it $L$. We shall try to give a meaning full geometric interpretation
to it.

\begin{equation}
\Lagr (q_i,\dot{q}_i,t) 
=\frac{m_1(\dot{x_1}^2+\dot{y_1}^2)+m_2(\dot{x_2}^2+\dot{y_2}^2)}{2}
+V(x_i, y_i)
\end{equation}

 The Lagrangian function has a singular term
 representing the infinitely strong wall and the perfectly
elastic collisions. Allow a lack of rigour here and let us
put this term as an infinite Heaviside function: 

\begin{equation}
u(x)=\begin{cases}
0 &\mbox{ if } x < 0 \\
\infty &\mbox{ if } x \ge 0 
\end{cases}
\end{equation}

By taking the polar coordinates in the usual way:
\begin{align}
x_i & = \rho_i \cos \theta_i, \\
y_i & = \rho_i \sin \theta_i, \\
\end{align}
the  potential could be written as the sum of the 
terms of each possible coalition:
\begin{equation}
V(x_i, y_i)=u(1-r-\rho_1)+u(1-r-\rho_2)
+u(-(\rho_1^2+\rho_2^2-\rho_1\rho_2\cos(\theta_1-\theta_2))).
\end{equation}
We should notice that, even when the difference in
angles appears, the sum of them does not. 

We can obtain the individual angular momenta $l_i$ as partial 
derivatives of the Lagrangian function with respect to
the $\dot{\theta_i}$. 

\begin{equation}
L=l_2+l_2=\frac{\partial \Lagr}{\partial \dot{\theta}_1}
+\frac{\partial \Lagr}{\partial \dot{\theta}_2}
\end{equation}
This turns to be, thank to the convenient linearity
of the partial derivative \emph{with respect to the deriving}:
\begin{equation}
\frac{\partial}{\partial \dot{\theta}_1}
+\frac{\partial}{\partial \dot{\theta}_2}=
\frac{\partial}{\partial(\dot{\theta}_1 +\dot{\theta}_2)},
\end{equation}
and also:
\begin{equation}
(\dot{\theta}_1 +\dot{\theta}_2)=\frac{d}{d t} (\theta_i +\theta_2).
\end{equation}

Then the total angular momentum $L$ is conjugate to the
sum of the angular variables $\theta_i$. This quantity is not as
intuitively clear as it seems. Let us try and find a coordinate
system where this quantity shows its conservation explicitly.
Remember that angles do not appear in the kinetic
therm in usual polar coordinates:

\begin{equation}
\begin{split}
T & =1/2 (m (\dot{x}^2+\dot{y}^2)) \\
 & =1/2 (m (\dot{\rho}^2+\rho^2 \dot{\theta}^2))
\end{split}
\end{equation}

Rewriting the Lagrangian function with this considerations in
polar coordinates, we can try to make obvious that
the \emph{individual angles} are not good coordinates
for the problem.

\begin{equation}
\begin{split}
\Lagr (q_i,\dot{q}_i,t) 
= & 1/2 (m_1 (\dot{\rho_1}^2+\rho_1^2 \dot{\theta_1}^2))
  +1/2 (m_2 (\dot{\rho_2}^2+\rho_1^2 \dot{\theta_2}^2)) \\
 & +u(1-r-\rho_1)+u(1-r-\rho_2) \\
 & +u(-(\rho_1^2+\rho_2^2-\rho_1\rho_2\cos(\theta_1-\theta_2))).
\end{split}
\end{equation}.

Then one can see that the partial derivatives with respect
to $\theta_i$ eliminate each other, making $L$ a conserved
quantity, and that the combination $\theta_1-\theta_2$,
the relative phase, appears as a relevant quantity. It
comes to mind the simple, linear change of variables:
\begin{align}
\Theta & = (\theta_1+\theta_2)/C \\
\vartheta  & = (\theta_1-\theta_2)/C,
\end{align}
where $C$ can be $1,2,\sqrt{2}$ according
to researchers taste and convenience. For graphical representation
purposes, here I use $C=2$. This makes the $\Theta$ variable
the angle exactly at the middle between the radial vectors
of both discs, and the little $\vartheta$ variable is just how
much each of the two disks depart from this middle value.

Expressing the Lagrangian and obtaining the momenta in terms of these
variables is an exercises in tedious substitutions:

\begin{align}
\dot{\theta_1} & = \dot{\Theta}+\dot{\vartheta}, \\
\dot{\theta_2} & = \dot{\Theta}-\dot{\vartheta}; \\
\dot{\theta_1}^2 & = \dot{\Theta}^2+\dot{\vartheta}^2+2\dot{\Theta}\dot{\vartheta}, \\
\dot{\theta_2}^2 & = \dot{\Theta}^2+\dot{\vartheta}^2-2\dot{\Theta}\dot{\vartheta}.
\end{align}



\begin{equation}
\begin{split}
\Lagr (q_i,\dot{q}_i,t) 
= & \frac{1}{2} \Big( (m_1\dot{\rho_1}^2+m_2\dot{\rho_2}^2)
  + m_1 \rho_1^2 (\dot{\Theta}^2+\dot{\vartheta}^2+2\dot{\Theta}\dot{\vartheta})
  + m_2 \rho_2^2 (\dot{\Theta}^2+\dot{\vartheta}^2-2\dot{\Theta}\dot{\vartheta})
\Big) \\
 & +u(1-r-\rho_1)+u(1-r-\rho_2) \\
 & +u(-(\rho_1^2+\rho_2^2-\rho_1\rho_2\cos(2\vartheta))).
\end{split}
\end{equation}.

There is no way to simplify further the kinetic therm in this
set of coordinates, and actually, is not warranted.
To get the new momenta, we work straight from the
definition and check with Noether's conserved quantity:

\begin{equation}
\begin{split}
L : & = \frac{\partial \Lagr}{\partial \dot{\Theta}} \\
&=m_1\rho_1^2\dot{\Theta}+m_1\rho_1^2\dot{\vartheta}
+m_2\rho_2^2\dot{\Theta}-m_2\rho_2^2\dot{\vartheta} \\
&=\frac{\dot{\theta_1}+\dot{\theta_2}}{2}m_1\rho_1^2+
\frac{\dot{\theta_1}-\dot{\theta_2}}{2}m_1\rho_1^2
\frac{\dot{\theta_1}+\dot{\theta_2}}{2}m_2\rho_2^2-
\frac{\dot{\theta_1}-\dot{\theta_2}}{2}m_2\rho_2^2 \\
&= \dot{\theta_1} m_1 \rho_1+\dot{\theta_2} m_2 \rho_2 \\
& = l_1+l_2.
\end{split}
\end{equation}

I noticed that there is a factor $2$ missing somewhere,
see if you can spot where it leaked  \ldots



\section{Some Keplerian Considerations}

If we had only one particle with billiard dynamics in a rotational
invariant system,  there is a trick which shows the irrelevant characteristic
of the variable conjugated to angular momenta \cite{Benet}.
We proceed to rotate using an ``effective'' angular velocity, which
renders the angular variable constant in such a representation.

A point particle, (or the centre of a disk, for all effects), bouncing
in a rigid wall describes chords of constant length at constant velocity,
which gives an average angular velocity. If we rotate the frame of
reference with this angular velocity smoothly, it will appear as if
the particle describes a bounce always on the same spot, describing
a cusped curve between bounces. The effective angular
velocity, $\omega$, has a relatively unappealing expression, but the
curve is quite well understood. 

The formulae look unappealing by the simple reason that
the particle is travelling in straight lines, and the angle sweeps
circular sectors of circle. So the homogeneous time using
spent in travelling the straights line looks weird from a curve 
standpoint. 
Let us calculate this effective angular velocity.
The geometry is as in figure \ref{geocaustica01}
The interval has length  $ 2 R \sin \theta$, so the
time in which is travelled is $ t= 2 R \sin \theta/\|v\|$.
Then the average angular velocity would be

\begin{figure}
\centering
\includegraphics[scale=0.88]{GeometriaCaustica01.pdf}
\caption{The geometry for  a single particle caustic in
the billiard.} \label{geocaustica01}
\end{figure}

\begin{equation}
  \bar{\omega}=\frac{2 \|v\| } { 2 R \sin \theta}.
\end{equation}

Now relating $\omega$ with the properties of the particle we 
could arrive at final expression:

\begin{equation}
\theta=\arccos \Big(\frac{L/\|p\|}{R}\Big)
\end{equation}

The effect of rotating the frame
by this constant angular velocity is illustrated on the figure
\ref{effectiveomega}.

In that last equation appears the quantity $L/\|p\|$, which
in the single particle billiard is the radius of the inner caustic.
This circular caustic is a configuration space effect of
super-integrability:
the conservation of angular momentum.
 
The angular momenta is not an angular velocity, except 
in the very special case that particles are forced to travel 
in perfect circles, and we measure the momentum from the 
centre of the circle.

The conservation of angular momenta is Kepler's second law:
Equal Areas are swept in equal time. So in this system
(or any rotational invariant system), the pertinent
quantity is the rate at which all the particles 
swap, all together, area. This area, is, of course,
directed, and we shall take into account the direction
of rotation to determine its sign. 

\begin{figure}[h]
  \centering
  \begin{subfigure}[b]{0.48\textwidth}
    \centering
    \includegraphics[scale=0.88]{TrayectoriaNORotada01.pdf}
    \caption{``Fixed Stars'' frame of reference.}
    \label{fixedtray01}
  \end{subfigure}%
\begin{subfigure}[b]{0.48\textwidth}
    \centering
          \includegraphics[scale=0.88]{TrayectoriaRotada01.pdf}
                \caption{Rotating frame of reference.}
                \label{ftigre01}
  \end{subfigure}
\caption{ With one single particle or disk, we can set a 
homogeneous rotating system of reference, which makes 
the conservation of angular momenta appear as an
effective conserved angular velocity. Sadly, this tricks
stops working easily if we have two interacting particles.
 }\label{effectiveomega}
\end{figure}

Taking that into account, it seems that this
average $\omega$ is not a very useful quantity to rotate the system
with two disks. Even if the disk do not interact, each one is hitting
the walls at a different rate, typically with an irrational ratio
and phase. 

We should thing in terms of the \emph{instantaneous} angular
velocity. Remember: the Keplerian property of the conservation
of angular momentum means not that ``angle sweeping'' is constant,
but ``area sweeping''. This still means that the angles are
``cyclic'' variables, which is not the same as saying that
they have constant linear change, but have regular, quasi-periodic
behaviour. This can be better understood with an angle vs angular
velocity plot, in the figure \ref{anglevsvelo}. For the
moment, we have made so that the second disk is static at the middle,
so its graph is just a point in the origin.
Notice, carefully, how each cusp in the figure
is evenly spaced from its neighbours. That should be obvious
from the dynamics: the particle or disk is bouncing around a 
caustic, always describing the same chords of the circle,
monotonically rotating. 

\begin{figure}[h]
\centering
\includegraphics[width=0.8\textwidth]{AnglevsVelocity01.png}
\caption{The cyclic behaviour of the angular variable.}
\label{anglevsvelo}
\end{figure}

Even when we do not have a clear geometric interpretation
for $\theta_1+\theta_2$, the conservation of the conjugate
momentum tells us that it is a cyclic variable.
Its behaviour should be as as the curves shown in
that plot, even when the two disks interact, and therefore,
change their behaviour in this space individually.
Making a quick numeric test (where errors in the
$2 \pi$ modulus are evident) we can appreciate some regularity
on that ``quasi-periodicity''. In the last figure, \ref{anglevsvelotest},
we can appreciate that even after a collision between the disks,
appreciable as a discontinuity in every line, the
interval covered by each black ``mountain'' remains constant.
We can test the exact value numerically after the weekend!


\begin{figure}[h]
\centering
\includegraphics[width=0.8\textwidth]{AnglevsVelocity02.png}
\caption{The cyclic behaviour of the angular variable, and their sum
as a black line.}
\label{anglevsvelotest}
\end{figure}

\bibliography{../../notasmixtas/TwoDiskBiblio}

\end{document}
