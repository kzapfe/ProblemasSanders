\documentclass[letterpaperr,12pt]{article}

\usepackage[utf8]{inputenc}
\usepackage{amsmath,amssymb}
\usepackage[spanish]{babel}

\usepackage[T1]{fontenc}  
\usepackage{textcomp}


%\usepackage[garamond,cmbraces]{newtxmath}

\usepackage{xspace}
\usepackage{graphicx}

\usepackage{float}
\usepackage{caption}
\usepackage{subcaption}
%\usepackage{anysize}
\usepackage[margin=2cm]{geometry}

%\usepackage{bm}

\bibliographystyle{alpha}

\newcommand{\Acase}{\textbf{A}\xspace}
\newcommand{\Bcase}{\textbf{B}\xspace}
\newcommand{\Ccase}{\textbf{C}\xspace}
\newcommand{\Dcase}{\textbf{D}\xspace}
\newcommand{\Uno}{\textbf{1}\xspace}
\newcommand{\Dos}{\textbf{2}\xspace}

\renewcommand{\rmdefault}{ugm}
  

\title{Aproximaciones Cilíndricas, Relación entre cantidades numéricas.}
\author{W. P. K. Zapfe}


\begin{document}

\maketitle


\section{Dos animálculos en un cuadrado: 4D efectivas.}

La generalización es menos fea de lo que uno podría pensar.
Establezcamos las relaciones que definen la geometría:

\begin{align}
x_1,x_2 \in  & [ -a, a], \\
y_1, y_2\in & [ -b, b], \\
(x_1-x_2)^2+(y_1-y_2)^2 & \ge (2 r)^2.
\end{align}

Como podemos ver, son casi iguales, excepto la auscencia de la
constricción $y_2=0$. La misma transformación aplicada a 
las coordendadas nos simplifica la última desigualdad
y nos define un círculo en el plano $x,y$. La frontera
absorvente se vuelve un cilindro cuadridimensiónal, con
dos ejes rectos y una base cilíndrica. Los ejes son a lo largo de
las coordenadas $X,Y$, y su longitud no depende una de la otra, siendo
cada uno de ellos diagonal a una de las caras del hipercubo. Asi
que su extensión, en ambos casos, despreciando las cuñas, es la
diagonal del cuadrado en las $x_i$ o en las $y_i$.

El volumen y el área totales resultan ser entonces, en
términos tanto del diametro como del radio;

\begin{align}
 V & = d^2(4\pi ab - [(a+b)8/3] d +d^2/2) \\
 & = r ^2(16\pi ab - [(a+b)64/3] r +8 r^2)  \\
A & = 8\pi\sqrt{2} a b d -8 \sqrt{2}(a+b) d^2 +4 d^3/\sqrt{2} \\
 & =  16\pi\sqrt{2} a b r  -32 \sqrt{2}(a+b) r^2 +16  \sqrt{2} r^3.
\end{align}

\subsection{El Hipercilíndro equivalente}

Ahora bien, tenemos dos ``alturas'' que ajustar, y el
 hipercubo
tiene una relación de aspecto de $a/b$. Quisieramos
preservar esto, asi que sugiero corregir las dos alturas
equivalentes del cilindro por el mismo factor $z$.

\begin{align}
V &= R^2 ( a z)( b z) \pi \\
A &= R ( a z)( b z) 2  \pi \\
\rightarrow & R= 2V/A, \text{ } z^2=A^2/( 4\pi V a b).  
\end{align}

Honestamente, poner esto en términos de $r_0$ está bastante feo, pero no imposible.
El radio del cilindro interior queda así:

\begin{equation}
R_{int}=\frac{ (2 \pi ab)r - [4/3 (a+b)]r^2 +8r^3}
{\sqrt{2}(\pi ab -2(a+b) r +2r^2)}
\end{equation}

El comportamiento aproximado para $r\ll a,b$ es $R_{int} \approx r\sqrt{2}$, lo cual
es lo correcto. 

Se puede ver que usando sólo los primeros términos arriba y abajo, 
recuperamos la expresión
que resultaría de no haber cuñas. 

El término correctivo $z$ adimensional esta dado por la expresión:

\begin{equation}
R_{int}=\frac{r_0 [ 4 \pi ab + r_0 ^2 (10 \pi -13/3) - 4 \sqrt{2}\pi r_0 (a+b)]}
{\sqrt{2}\pi[2 ab- 2r_0 \sqrt{2} (a+b)] +r_0^2 (8\sqrt{2} \pi + \pi/2 -3/2)}
\end{equation}

\begin{equation}
z^2=\frac{16 \big( ab\pi -2(a+b)r+2r^2 \big)^2}
{ a b \pi (2\pi ab -[8(a+b)/3]r+r^2}
\end{equation}
También en el límite $r\ll a,b$ da un resultado coherente: $z\approx 2\sqrt{2}$,
que sería el factor de escala para la convención geométrica usada despreciando
las cuñas.

Una vez más, el cilindro exterior tendrá las mismas alturas que el interior,
pero el radio estará determinado por el volumen total:
\begin{equation}
R_{ext}^2=\frac{16ab}{\pi z^2},
\end{equation}
que también da el resultado correcto si $r\ll a,b$.


\section{Formulazo de Szabo}

En la fórmula de Szabo, para el caso con interior completamente
absorvente, la cantidad relevante que aparece es $x=R_{int}/R_{ext}$.
Usando los resultados de arriba, podemos llegar a que está cantidad está
dada por
\begin{equation}
x=\frac{\sqrt{V}}{4ab}.
\end{equation}
Dado que por la simetría translacional del sistema equivalente
esto se puede reducir a un problema en dos dimensiones, tenemos
que el tiempo de absorción promedio es, segun Szabo
\begin{equation}
\tau(x)=\frac{D}{R_{ext}}
\bigg[ 
\frac{x^2-3}{8}-\frac{\ln (x)}{2(1-x^2)}\bigg] 
\end{equation}

Esto reduciría todo a hacer simplemente el ajuste de la $D$ efectiva.


\end{document}
