\documentclass[letterpaperr,12pt]{article}

\usepackage[utf8]{inputenc}
\usepackage{amsmath,amssymb}


\usepackage[T1]{fontenc}  
\usepackage{textcomp}


%\usepackage[garamond,cmbraces]{newtxmath}

\usepackage{xspace}
\usepackage{graphicx}

\usepackage{float}
\usepackage{caption}
\usepackage{subcaption}
%\usepackage{anysize}
\usepackage[margin=2cm]{geometry}

%\usepackage{bm}

\bibliographystyle{alpha}

\newcommand{\Acase}{\textbf{A}\xspace}
\newcommand{\Bcase}{\textbf{B}\xspace}
\newcommand{\Ccase}{\textbf{C}\xspace}
\newcommand{\Dcase}{\textbf{D}\xspace}
\newcommand{\Uno}{\textbf{1}\xspace}
\newcommand{\Dos}{\textbf{2}\xspace}

\renewcommand{\rmdefault}{ugm}
  

\title{Territorial spreading of epidemics}
\author{W. P. K. Zapfe}


\begin{document}


\maketitle

\begin{abstract}


\end{abstract}


\section{Introduction}


\section{The Model}

For our purpose, the \emph{land} is just the euclidian plane without any
special constructs. Every \emph{animal} shall be represented by a circle,
which is equivalent to its radius of interaction. All the animals
have the same radius of interaction, $\rho_0$, and animals at a distance
less than $2 \rho_0$ from each other are prone to contagion. A animal shall have
a square territory in which it moves randomly. The territories of the animals
can overlap, but not to much, each animal has a \emph{private} area and 
a \emph{shared} area. Areas shared between three or more animals should be rare,
making extremely rare triple encounters.  

Each animal shall be caracterized by its Cartesian coordinates and a boolean
variable indicating if its \emph{sick} or not. The territories of all infected
animals is the quarantine area. We are manly interested in the grow of
this area as a function of time for various cases. We shall start
thinking that the rate of recovery or death is irrelevant in the time scales
considered for infection.
 
The rate of spread of the disease, measured as change of quarantined area
per unit time depends strongly on the \emph{first encounter time} for
two animals which have overlapping territories. This problem is very
difficult to solve analitically, as it results in a difusive equation
in a very inconvenient geometry. We shall investigate numerically
this parameter as function of the \emph{mean} overlapping area for a large
number of animals, and then compare it with the spreading of the disease.

At first, only one animal shall be sick. The centers of the territories
shall be close to a regular cuadrangular grid, with random displacements
from it, picked from a Gaussian distribution. The variance of such a
distribution is responsible for the amount of average overlap. 

\section{Previous results}

A one dimensional  simplified version of this problem was researched 
in \ref{SandersLuca}. There was no need for the radius of interaction,
and a rough ``area preserving'' approximation to the dffusive equation
which gives the first encounter times and the mean times could be solved.
We can gain gain insight explaining the very dificult task of generalizing
this result. 



\end{document}

