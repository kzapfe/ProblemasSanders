\documentclass[letterpaperr,12pt]{article}

\usepackage[utf8]{inputenc}
\usepackage{amsmath,amssymb}

\usepackage[spanish]{babel}
\usepackage[utf8]{inputenc}  
\usepackage{textcomp}


%\usepackage[garamond,cmbraces]{newtxmath}

\usepackage{xspace}
\usepackage{graphicx}

\usepackage{float}
\usepackage{caption}
\usepackage{subcaption}
%\usepackage{anysize}
\usepackage[margin=2.54cm]{geometry}

%\usepackage{bm}

\bibliographystyle{alpha}

\newcommand{\Acase}{\textbf{A}\xspace}
\newcommand{\Bcase}{\textbf{B}\xspace}
\newcommand{\Ccase}{\textbf{C}\xspace}
\newcommand{\Dcase}{\textbf{D}\xspace}
\newcommand{\Uno}{\textbf{1}\xspace}
\newcommand{\Dos}{\textbf{2}\xspace}

\renewcommand{\rmdefault}{ugm}
  

\title{Tiempos de Primer Encuentro en Caminantes Aleatorios en $2D$}
\author{W. P. K. Zapfe}


\begin{document}

\maketitle

\section{Introducción}

En este trabajo hemos investigado mecanísmos de primeros encuentros
para caminantes aleatorios que se encuentran inmersos en espacios
planos bidimensionales. La motivación original del proyecto es encontrar
un modelo de primeros principios que simule la propagación de enfermedades
en animales altamente territoriales, aunque el modelo puede servir
para otros problemas donde se requiere estudiar la velocidad
de algun frente de propagación de una cualidad en un medio con ciertas
restricciones a la movilidad de los agentes. Actualmente existen
modelos de campo medio que dan buenos resultados sobre el número
de individuos afectados por la enfermedad o la cualidad, sin embargo
hay pocos resultados rigurosos sobre el área infectada, notablemente
aquí destaca el trabajo previo del Dr. Sanders y colaboradores 
en espacios unidimensionales \cite{SandersLuca}.
El propósito del presente trabajo es poder extender dichos resultados
a dos (y eventualmente más) dimensiones. En este reporte de investigación
describo primordialmente mi contribución al proyecto, excepto
donde se indique lo contrario. 

\section{Consideraciones sobre la dimensión}

La primera contribución que realicé al proyecto fue el mero señalamiento
de la imposibilidad de extender \emph{directamente} el modelo anterior
basado en caminantes en celdas discretas sin tomar en cuenta
un parámetro más que no existiá en el caso anterior. Este
razonamiento proviene de la \emph{dimensionalidad} de los objetos, 
relativa al espacio donde se encuentran.
Una mera división en celdas de territorios bidimensionales con
caminantes que ocupen una celda a la vez, llevado al  límite continuo,
produce dos objetos puntuales que no tienen posibilidad alguna 
de encontrarse. Así 
que se añadió un parámetro extra a los caminantes: un radio de interacción.
Esto hace que la sección del espacio de configuraciones que
representa un encuentro entre los dos caminantes tenga codimensión
uno. Para aclarar eso tomamos una versión reducida del problema,
donde uno de los caminantes sólo se moviera a lo largo de una
linea horizontal. Le llamaremos la versión $3D$ del problema. Construí
unas imágenes del espacio de configuraciones del espacio de configuración
total para que se pueda apreciar la dificultad de la dimensión de los
objetivos, que muestro en la figura \ref{celdas3d}. 


\begin{figure}[h]
  \centering
  \begin{subfigure}[b]{0.49\textwidth}
    \centering
    \includegraphics[width=0.99\textwidth]{Celdas_7_con_encuentro.pdf}
    \caption{7 celdas por lado.}
    \label{gruesaceldas}
  \end{subfigure}
  \begin{subfigure}[b]{0.49\textwidth}
    \centering
    \includegraphics[width=0.99\textwidth]{Celdas_20_con_encuentro.pdf}
    \caption{20 celdas por lado}
    \label{finaceldas}
  \end{subfigure}
  \caption{La versión $3D$ del problema, con celdas finitas. 
La condición de encuentro es el conjunto de celdas azules, cuyo
número crece linealmente inversamente proporcional al lado de la celda pero
cuyo volumen tiende a cero como el cubo de esta razón. Eventualmente 
su límite es un conjunto de medida cero.}\label{celdas3d}
\end{figure}


\begin{figure}[h]
  \centering
   \includegraphics[width=0.45\textwidth]{Celdas_15_con_radio.pdf}
   \caption{La versión $3D$ del problema, con celdas finitas, pero
con un radio de interacción mayor a cero. 
La condición de encuentro es el conjunto de celdas azules, }\label{celdas3dconradio}
\end{figure}


\section{La Geometría del Problema}

Nuestra intención original era verificar la posible aplicabilidad de la
fórmula de Szabo \cite{Szabo80} y los resultados de Sanders \cite{SandersLuca}
a este mayor número de dimensiones. La fórmula es adecuada si la sección
relevante del problema se puede aproximar por un anillo, 
siendo el resto de las dimensiones del espacio efectivamente 
irrelevantas. Para eso habría primero que considerar cual es 
la geometría real del espacio del problema y como podríamos simplificarlo
hasta que tuvieramos algo parecido a una sección anular.

La verdadera geometría del problema en $4D$ es un prisma rectangular
atravezado diagonalmente por un cilindro con dos ejes rectos,
lo cual daba la esperanza de que las aproximaciones mencionadas
funcionaran. La condición de encuentro es que los caminantes aleatorios
se encuentren a menos distancia que la distancia de interacción. Esto
produce una sóla condición sobre las coordenadas libres del sistema, 
expresable por una simple desigualdad:
\begin{equation}
(x_1-x_2)^2+(y_1-y_2)\ge (2r)^2
\end{equation}
Un simple cambio lineal de coordenadas hace ver que
esto excluye un corte circular. Esto nos dió esperanzas para
intentar aplicar los resultados mencionados. El caso más general
(territorios diferentes para los ``animales'') resulta,
infelizmente, en condiciones de frontera para el primero
impacto demasiado complicadas, así que concentramos los primeros
esfuerzos analíticos sobre el caso en que los
dos animales compartieran el mismo territorio. 
Esto resulta en un espacio de configuración total cuya espacio
libre es un prisma rectangular atravezado diagonalmente
por un cilindro, tanto en $3D$ como en $4D$. Los parámetros
que suponemos adecuados para aproximar todo por dos
cilindros coaxiales (el \emph{objetivo} y el \emph{espacio libre})
son los volúmenes y las áreas. Heurísticamente suponemos que el 
volumen libre es el parámetro fundamental para aplicar 
las fórmulas conocidas. Hemos recalculado entonces
los volúmenes y áreas de estos objetos geométricos y verificado
contra resultados anteriores en caso de existir \cite{Munakata02},
y en caso contrario comparado contra nuestras propias 
simulaciones numéricas.


El resultado de este esfuerzo es un conocimiento muy extenso
de las caracterísitcas geométricas y topológicas del sistema. Sabemos
que efectivamente podemos hacer cortes transversales sobre el espacio
de configuración total, de forma que el espacio libre sea topologicamente un
anillo. Sin embargo esto no ha sido suficiente para garantizar
el funcionamiento de las fórmulas que pretendiamos usar.
En el caso de que hubieramos tenido sólamente un anillo bidimensional
efectivo como espacio del problema, las fórmulas de tiempo medio
de Szabo, Schultz y Schultz dependerían sólamente de la proporción
entre el radio interior y exterior del anillo, de la forma siguiente:
\begin{equation}\label{Szabo01}
\langle t \rangle \frac{D}{R^2}=
\bigl((r/R)^2-3\bigr)/8 -\frac{\ln (r/R)}{2(1-(r/R)^2)},
\end{equation}
donde $D$ sería una constante difusiva efectiva.
El problema es encontrar un buen valor para dichos radios, que no corresponden
a los radios de interacción de los caminantes, sino dependen de la aproximación
que hayamos usado para reducir el problema. Dados los resultados anteriores,
suponemos confiadamente que la mejor aproximación que podemos llevar a cabo
resulta de preservar el volumen libre y el área de \emph{de contacto},
es decir, el área del cilindro que representa la condición de encuentro.
Sin embargo, como se muestra en las figuras de la siguiente
sección, esto parece no haber sido suficiente para dar un buen resultados. 
Los resultados numéricos fueron el detonador de una busqueda más extensa.


\section{El Esfuerzo numérico}

La parte primordial de la investigación está fundamentada en las 
simulaciones computacionales, que de ser correctamente llevadas a cabo,
validarían nuestras conjeturas sobre los tiempos de primer encuentro.
Para ello tuve que escribir, probar, corregir y paralelizar
los códigos que llevaron a cabo las simulaciones. Esto fue enriquecido
por programas realizados para producir animaciones en tiempo real
y poder visualizar tanto los problemas fundamentales como el problema
de motivación original. Parte de la numérica también fue resuelta
con técnicas de álgebra lineal por el Dr. Sanders, produciendo
lo que llamamos ``resultados numéricos exactos'', contra los
cuales es posible verificar las simulaciones. También llevamos a cabo
simulaciones con dinámicas distintas, para excluir la posibilidad de
que los detalles exactos de la mecánica influyeran sobre los resultados numéricos.
Los resultados se muestran resumidos en la figura \ref{Resultados01}.

\begin{figure}[h]
  \centering
   \includegraphics[width=0.95\textwidth]{KarelDavidSzabo01.pdf}
   \caption{Resultados numéricos contra la formula de Szabo, obtenida
a partir de la preservación del volumen. Es claro que esta
aproximación resulta muy pobre, excepto para el rango de valores
de rango de interacción muy pequeños. }\label{Resultados01}
\end{figure}

Otro argumento heurístico consiste en tomar en cuenta el decrecimiento 
del volumen libre disponible a lo largo del eje o los ejes del
cilindro. En ese caso la aproximación trabajada resultaría muy pobre.
Para poner aprueba esta conjetura me dediqué a simular un sistema
difusivo en cuatro dimensiones, cuyo espacio libre fuera cónico y 
su objetivo absorvente un cilindro coaxial. Los resultados se
muestran en la figura \ref{Conos01}.

\begin{figure}[h]
  \centering
   \includegraphics[width=0.95\textwidth]{KarelDavidSzaboConos02.pdf}
   \caption{Resultados numéricos del sistema $4D$ contra la 
simulación de un sistema difusivo en 4 dimensiones con espacio
libre cónico y un absorvente cilíndrico coaxial. Se puede
apreciar que el comportamiento es el mismo en esta primera prueba.
 }\label{Conos01}
\end{figure}

La última figura soporta el argumento de que la geometría
exacta del espacio de configuración no es aproximable por cilindros,
y que el decrecimiento del espacio libre a lo largo de los ejes
de simetría no resulta despreciable. Por ende, nuestro esfuerzo
por aplicar la fórmula de Szabo directamente debe ser modificado para
tomar en cuenta la geometría del problema.

\section{La Fórmula de Fick-Jacobs}

Finalmente, dada la difucultad que hemos enfrentado para encontrar la
solución de los primeros encuentros del problema, hemos decidido buscar
otra posible ruta, que es la llamada fórmula Fick-Jacobs, usada
exitosamente en otros problemas de difusión para encontrar primeros
tiempos de encuentro en sistemas con absorventes dispuestos a lo largo
de un eje \cite{Jacobs}. La fórmula fue desarrollada para cubrir casos
unidimensionales, pero ha sido ya aplicada a casos de mayor dimensión, 
considerando el espacio extra como un efecto de concentración \cite{Dagdug}
 Aún no sabemos
si el grado de aplicabilidad de ésta funcionará para las geometrías que nosotros
estamos considerando, pero ya contamos con un camino delineado para
provar su posible aplicación.
Siguiendo el trabajo de Vazquéz y Dagdug, tenemos que representar
nuestro sistema por un canal unidimensional efectivo. Dados los resultados presentados
en la figura, \ref{Conos01}, tomamos la aproximación cónica de la geometría como
un buen modelo para el espacio libre. Entonces podremos
considerar la dirección radial como la dirección efectiva del problema
unidimensional segun la fórmula Fick-Jacobs. De esta forma el 
objetivo absorvente quedaría ``en un extremo'' de la dirección efectiva,
el radio de interaccíon provería un final abrupto de esta curva
de concentración (véase la figura \ref{VolRadial}).

\begin{figure}[h]
  \centering
   \includegraphics[width=0.75\textwidth]{DiffVolPorRadio01.pdf}
   \caption{La diferencial de volumen (4D) como función del radio. Nuestra
concentración efectiva a lo largo de 
una sóla dimensión radial 
será proporcional a esta cantidad. 
El absorvente se encontrará en algun lugar del eje radial, dependiendo
del radio de interacción; estando
el espacio libre a la derecha. Las unidades son arbitrarias,
 }\label{VolRadial}
\end{figure}


Dicha fórmula fue desarrollada con la intención de resolver tiempos
de pasaje en sistemas difusivos cuyos objetivos (absorventes) estuvieran
en los extremos de una dirección dada. Nosotros tenemos una geometría donde
el absorvente esta a lo largo de un eje de simetría rotacional. 
Podríamos considerar el cambio de coordenadas polar como una
afectación propia de un campo de concentraciones no homogeneo. 
Tendremos que ver si la formula resultante de esta transformación funciona
al menos en una aproximación burda.

\bibliography{../../notasmixtas/General}


\end{document}