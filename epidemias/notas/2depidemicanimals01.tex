\documentclass[letterpaperr,12pt]{article}

\usepackage[utf8]{inputenc}
\usepackage{amsmath,amssymb}


\usepackage[T1]{fontenc}  
\usepackage{textcomp}


%\usepackage[garamond,cmbraces]{newtxmath}

\usepackage{xspace}
\usepackage{graphicx}

\usepackage{float}
\usepackage{caption}
\usepackage{subcaption}
%\usepackage{anysize}
\usepackage[margin=2cm]{geometry}

%\usepackage{bm}

\bibliographystyle{alpha}

\newcommand{\Acase}{\textbf{A}\xspace}
\newcommand{\Bcase}{\textbf{B}\xspace}
\newcommand{\Ccase}{\textbf{C}\xspace}
\newcommand{\Dcase}{\textbf{D}\xspace}
\newcommand{\Uno}{\textbf{1}\xspace}
\newcommand{\Dos}{\textbf{2}\xspace}

\renewcommand{\rmdefault}{ugm}
  

\title{Epidemic spread in territorial Animals}
\author{ D. P. Sanders, W. P. K. Zapfe}


\begin{document}

%r\marginsize{2cm}{2cm>}{2cm}{2cm}
\maketitle

\begin{abstract}

We will expand previous work by Sanders et Luca to animals
moving in a two dimensional territory. A lot of simplifications
are made in order to bring this toy model to something
tractable.

\end{abstract}


\section{Definitions}

We start by possing an array of ``animals'' on the plane.
each animal is confined to a territory, which we model by
a square of side one. The squares are centered approximatively
in an square lattice. Each animal moves as a random walker
inside its territory, and they can meet on overlaping territories.
If the animals where points, occupying exactly one cell
of the lattice in which they move, their chance
of encountering on the diffusive limit would be exactly zero.
We solve that giving the animals a radious of
interraction, which should be small compared to territory
and relative big compared to their stepsize. Animals
which are less than two radious of interaction apart can
be prone to infection from each other. 

Let us intrudece some notation and specifications of our
simulation. Each animal shall be denoted by an $a_k$
 symbol, the $k$ serving as a counting index, and its
square territory shall be denoted by $T_k$. Each 
territory is centered at a point $(\bar{x}_k, \bar{y}_k)$,
which is inmutable during the ``life'' of its owner. 
The centre of each territory is randomly displaced
from the points of a square lattice. The interesting
parameter in this case is a number denoted $h$ which
represents how territorial the species is. The $h$ is the
spacing in the lattice. A smaller $h$ results
in territories that have large area of overlapping,
producing really fast infection rates, but an $h$ larger
than one produces very disconected territories,
and thus very low infection rates. The selection
of each centre of the territory is a Gaussian displacement
from a lattice of spacing $h$. That gives for
$z=x,y$ the next expression:

\begin{gather}
\bar{z}= j \cdot h + \epsilon, j\in \mathbb{N},\\
P(\epsilon)=\frac{\exp(-\epsilon^2/\sigma^2)}{\sigma\sqrt{\pi}}, \\
\sigma=h/4.
\end{gather}

The variance in the pdf for the displacement is
adjusted acording to $h$, so that the territories keep
a private area. The distribution of territories looks
locally as in figure \ref{territories}.
In the case that $h > 1.0$ the animal territories
are largerly disconected, thus making epidemic spread
imposible. We focus our interest in  $h<1.0$.

All animals will have the same radius of interaction, $\rho$.

\begin{figure}{h}
 \centering
  \begin{subfigure}[b]{0.4\textwidth}
    \centering
    \includegraphics[width=0.99\textwidth]{Territorios05.pdf}
    \caption{$h=0.5$}
    \label{h05}
  \end{subfigure}
  \begin{subfigure}[b]{0.40\textwidth}
    \centering
    \includegraphics[width=0.99\textwidth]{Territorios075.pdf}
    \caption{$h=0.75$}
    \label{h075}
  \end{subfigure}
 \\ 
 \centering
  \begin{subfigure}[b]{0.40\textwidth}
    \centering
    \includegraphics[width=0.99\textwidth]{Territorios090.pdf}
    \caption{$h=0.5$}
    \label{h09}
  \end{subfigure}
  \begin{subfigure}[b]{0.40\textwidth}
    \centering
    \includegraphics[width=0.99\textwidth]{Territorios120.pdf}
    \caption{$h=1.20$}
    \label{h075}
  \end{subfigure}
  \caption{ The spread and overlaping of the square
territories, in function of $h$. Notice that
in the last figure the animals live in practical isolation,
making epidemic diseases by direct contact imposible. A single
territory has been remarked for comparition.}\label{territories}
\end{figure}

In our first approach, the probabiliy of infection
on contact is certain. 


\section{Random Walker considerations}


We pose
our reasoning by dividing the territory each animal into equal 
square cells. In the limit of very small cells, 
it shall behave as a diffusive particle with reflecting boundaries
inside each square. This continous approximation gives us a
induced distribution over the overlapping areas. The problem
of solving the first encounter time analitically there
is quite difficult, as the geometric conditions result in
non-separable parcial differencial equations \cite{NotasAnteriores}. 
Also, there are doubts in regards that a Random Walker is
a good aproximation the movements of a territorial animal,
although our model could be applied to bacteries in a jelly
suspension.

It should be possible to infer this first passage time
from the expantion of the epidemic, which intuitivelly should
be a simple algebraic law of the overlapping area. An educated
guess would be that is simply inverselly proportional to
the probability that both animals are in the overlapping
area, and proportional to the radius of interaction.




\end{document}

