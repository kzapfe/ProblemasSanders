\documentclass[letterpaperr,12pt]{article}

\usepackage[utf8]{inputenc}
\usepackage{amsmath,amssymb}
\usepackage[spanish]{babel}

\usepackage[T1]{fontenc}  
\usepackage{textcomp}


%\usepackage[garamond,cmbraces]{newtxmath}

\usepackage{xspace}
\usepackage{graphicx}

\usepackage{float}
\usepackage{caption}
\usepackage{subcaption}
%\usepackage{anysize}
\usepackage[margin=2cm]{geometry}

%\usepackage{bm}

\bibliographystyle{alpha}

\newcommand{\Acase}{\textbf{A}\xspace}
\newcommand{\Bcase}{\textbf{B}\xspace}
\newcommand{\Ccase}{\textbf{C}\xspace}
\newcommand{\Dcase}{\textbf{D}\xspace}
\newcommand{\Uno}{\textbf{1}\xspace}
\newcommand{\Dos}{\textbf{2}\xspace}

\renewcommand{\rmdefault}{ugm}
  

\title{Aproximaciones Cilíndricas, Relación entre cantidades numéricas.}
\author{W. P. K. Zapfe}


\begin{document}

\maketitle

\section{Un Cuadrado y una Linea}

\bf{En esta segunda versión, te paso las formulas con la convención
de la Distancia de interacción total, como te gusta.}

$R_0=2r_0$.

Primero establezcamos el caso más simple. El animal \Uno puede moverse
en un cuadrado de lado uno, mientras que el otro sólo se mueve en una linea
horizontal en el centro del cuadrado. Para simplificar cálculos
pongamos el origen de coordenadas en el centro del cuadrado.
Con esto las ecuaciones que definen el espacio disponible para
moverse libremente sin interacción son las siguiente:

\pmb{Usemos medidas generales para le retanugulo}.

\begin{align}
x_1,x_2 \in  & [ -a/2, a/2], \\
y_1\in & [ -b/2, b/2], \\
y_2 = & 0, \\
(x_1-x_2)^2+(y_1-y_2)^2 & \ge (2 r_0)^2,
\end{align}
usando $r_0$ como el radio de contacto entre animálculos.

Dado que podemos pensar en un radio total de interacción 
$R_0=2r_0$, escribiremos la última restricción en términos de éste.

\begin{equation}
(x_1-x_2)^2+(y_1-y_2)^2 & \ge (R_0)^2,
\end{equation}
 
A pesar de la restricción de $y_2$, es más cómodo tomarla en cuenta
desde un inicio para los cálculos.

El Volumen del cilindro ``de interacción'' (es decir, el cilindro al
interior del cubo) esta determinado por el la parte central
del cilindro y dos cuñas. El cambio de coordendas obtenido
mediante una rotación simple:

\begin{equation}\label{rotacion}
  \begin{pmatrix}
    a \\
    A
  \end{pmatrix}
  =\frac{1}{\sqrt{2}}
  \begin{pmatrix}
    1 & -1 \\
    1 & 1
  \end{pmatrix}
  \begin{pmatrix}
    a_1 \\
    a_2
  \end{pmatrix},
\end{equation}

Esto determina el radio del cilindro:
\begin{equation}
2x^2+2y^2 \ge (R_0)^2
\end{equation}

Resultando ser, entonces, $\sqrt{2} r_0$.
La parte sin cuñas del cilindro está delimita por la altura
``libre'', es decir, la que no está cerca de las esquinas del cubo. 
Esta resulta ser (un diagrama simple lo demuestra):

\begin{equation}
h_0=\sqrt{2} (a-R_0).
\end{equation} 

El resto del volumen es dos veces el volumen de una cuña de dos caras,
cada uno resulta ser (ver mis notas de la red de cilindros):

\begin{equation}
V_c=r_0^3 (2\pi-2).
\end{equation}

Entonces el volumen del cilíndro interior, contando cuñas y todo, es de:

\begin{align}
V_0 &= \sqrt{2} \pi (\sqrt{2} r_0)^2 (a-2 r_0)+ 2r_0^3 (2\pi-2)\\
&= 2  r_0^2 [\pi \sqrt{2} \pi (a - 2 r_0) + r_0 (2\pi-2) ] \\
&= 2  r_0^2 [ \pi \sqrt{2} a  + r_0 (2\pi-2-2 \sqrt{2}) ] 
\end{align}


El último renglón muestra la separación entre la parte
``cilíndrica'' y las cuñas.

Por cierto, como podemos ajustar dos parámetros para obtener
un cilindro del mismo volumen (radio y altura), tal vez podamos
también preservar el área en nuestro sistema simplificado.
El área de interacción del cilindro está dada por:

\begin{align}
A_0 &= 2\pi 2 r_0 (a-2 r_0)+ 2 r_0^2(\pi/2-1)\\
&= 2 r_0 [2\pi a +  r_0 (\pi/2-2\pi\sqrt{2}-1)]\\
\end{align}

Como podemos ver, las correcciones debidas a las cuñas son
de un orden superior siempre a la parte del cilindro convencional en
la variable $r_0$, por ende, mientras menor sea $r_0$, la correcion 
es menos importante. Asimismo, para $r_0 \approx 1/2$, el par de cuñas ocupan
casi todo el interior del cubo. 

Escogamos ahora un cilindro de radio $r_1$ y altura $h_1$ tales
que preserven el volumen y área del cilindro interior con cuñas.

\begin{align}
A_0 & = 2\pi r_1 h_1, \text{ } V_0 = \pi r_1^2 h_1 \\
\Rightarrow r_1 & = 2V_0/A_0, \text{ }h_1  =  A_0^2/(4 \pi V_0 ).  
\end{align}

Haciendo la talachita, podemos obtener una espantosa expresión adecuada:

\begin{align}
 r_1 & = \frac{2 r_0 [\sqrt{2} \pi a + r_0(2 \pi - 2 - 2 \sqrt{2})]}
 {[2 \pi a + r_0 (\pi/2 - 1 - 2\pi \sqrt{2})] } \\
h_1 & = \frac{1}{2\pi}
\frac{[2\pi a + r_0 (\pi/2 -2 \sqrt{2}\pi-1)]^2 }
{\sqrt{2} \pi a + r_0(2 \pi -2 - 2 \sqrt{2}) }
\end{align}

Uno puede ver que despreciando las contribuciones de la cuña se recupera
la identidad, $r_1= \sqrt{2} r_0, h_1=h_0=\sqrt{2} a $. 
Ahora, el cilindro que representará 
al cubo tendrá la misma altura $h_1$, pero también queremos preservar
su volumen, que es $1$, entonces el radio externo del sistema será
$R_1=\sqrt{a ^2 b /(\pi h_1)}$. 

\section{Dos animálculos en un cuadrado}

La generalización es menos fea de lo que uno podría pensar.
Establezcamos las relaciones que definen la geometría:

\begin{align}
x_1,x_2 \in  & [ -a/2, a/2], \\
y_1, y_2\in & [ -b/2, b/2], \\
(x_1-x_2)^2+(y_1-y_2)^2 & \ge (2 r_0)^2.
\end{align}

Como podemos ver, son casi iguales, excepto la auscencia de la
constricción $y_2=0$. La misma transformación aplicada a 
las coordendadas nos simplifica la última desigualdad
y nos define un círculo en el plano $x,y$. La frontera
absorvente se vuelve un cilindro cuadridimensiónal, con
dos ejes rectos y una base cilíndrica. Los ejes son a lo largo de
las coordenadas $X,Y$, y su longitud no depende una de la otra, siendo
cada uno de ellos diagonal a una de las caras del hipercubo. Asi
que su extensión, en ambos casos, despreciando las cuñas, es la
diagonal del cuadrado en las $x_i$ o en las $y_i$.
Entonces el volumen de la
parte central del hipercilíndro será simplemente
$V=\pi (\sqrt{2} r_0)^2 (\sqrt{2}a-2 r_0) (\sqrt{2}b-2 r_0) $.

La contribución de las cuñas es sólo un poco más tediosa.

\subsection{Cuñas cilíndricas en cuatro dimensiones}

Escencialmente la condición geométrica que describe
las cuñas en cuatro dimensiones es entéramente análoga
a la que la describe en tres. Además, las condiciones
en las $y_i$ y las $x_i$ son independenientes funcionalmente,
lo que hace que la cuña esté limitada en sus ``alturas'' 
de forma separada, por lo tanto, usando el mismo truco que en 
los billares cilíndricos, calculemos primero un cuarto de cuña.

\begin{equation}
V_{4c}=\iiiint dz dw \rho d \rho d \theta
\end{equation}

La integral sobre $z$ va de 0 a $r_0-x=r_0-\rho \cos \theta$ y 
la integral sobre $w$ de 0 a $r_0-y=r_0-\rho \cos \theta$.
La integral sobre $\theta$ de 0 a $\pi/2$ y 
la de $\rho$ de 0 a $r_0$.
Nótese que los parámetros $a,b$ no aparecen en ningun lado, 
ya que no importa dónde está la cuña, tiene el mismo volumen,
y $a,b $ sólo controlan su localización espacial. 

\begin{align}
V_{4c} & =\iint (r_0-\rho \cos \theta)(r_0 - \rho \sin \theta)
 \rho d \rho d \theta \\
&=\int (r_0^2 \pi/2 -2 r_0 \rho +2 \rho^2) \rho d \rho \\
&= r_0^4 (\pi/4-13/24).
\end{align}

Eso fue un cuarto de cuña. La cuña total nos da:
\begin{equation}
V_c= r_0^4 (\pi-13/6).
\end{equation}

El área se obtiene de forma análoga, obteniendo:

\begin{equation}
A_c= r_0^3 (\pi/2-3/2).
\end{equation}

Las unidades son las adecuadas, el volumen es de cuatro
dimensiones, y el ``área'' tiene tres.
El volumen y el área totales resultan ser entonces:

\begin{align}
 V & = r_0^2 [2 \pi  (\sqrt{2}a-2 r_0) (\sqrt{2}b-2 r_0) + r_0^2 (2\pi-13/3)] \\
 & = r_0^2 [4 \pi ab   + r_0^2 (10\pi-13/3)-4 \sqrt{2} \pi r_0 (a+b)] \\
A & = 4\pi r_0 (\sqrt{2}a-2 r_0) (\sqrt{2}b-2 r_0) +  r_0^3 (\pi/2-3/2) \\
 & = 2 \sqrt{2} \pi r_0 [2 ab -2\sqrt{2} r_0 (a+b)] +  r_0^3 (\pi/2+8\sqrt{2}\pi-3/2) 
\end{align}r

\subsection{El Hipercilíndro equivalente}

Ahora bien, tenemos dos ``alturas'' que ajustar, y el
 hipercubo
tiene una relación de aspecto de $a/b$. Quisieramos
preservar esto, asi que sugiero corregir las dos alturas
equivalentes del cilindro por el mismo factor $z$.


\begin{align}
V &= R^2 (\sqrt{2} a z)(\sqrt{2} b z) \pi \\
A &= R (\sqrt{2} a z)(\sqrt{2} b z) 2  \pi \\
\rightarrow & R= 2V/A, \text{ } z^2=A^2/( 8\pi V a b).  
\end{align}

Honestamente, poner esto en términos de $r_0$ está bastante feo, pero no imposible.
El radio del cilindro interior queda así:

\begin{equation}
R_{int}=\frac{r_0 [ 4 \pi ab + r_0 ^2 (10 \pi -13/3) - 4\sqrt{2} \pi r_0 (a+b)]}
{\sqrt{2}\pi[2 ab- 2\sqrt{2} r_0 (a+b)] +r_0^2 (8\sqrt{2} \pi + \pi/2 -3/2)}
\end{equation}

Se puede ver que usando sólo los primeros términos arriba y abajo, 
recuperamos la expresión
que resultaría de no haber cuñas. 

El término correctivo $z$ adimensional esta dado por la expresión:

\begin{equation}
R_{int}=\frac{r_0 [ 4 \pi ab + r_0 ^2 (10 \pi -13/3) - 4 \sqrt{2}\pi r_0 (a+b)]}
{\sqrt{2}\pi[2 ab- 2r_0 \sqrt{2} (a+b)] +r_0^2 (8\sqrt{2} \pi + \pi/2 -3/2)}
\end{equation}

\begin{equation}
z=\sqrt{
\frac{\big( 2\sqrt{2}\pi[ 2 ab- 2r_0 \sqrt{2}(a+b)] +r_0^2 (8\sqrt{2} \pi + \pi/2 -3/2) \big)^2}
{8 \pi ab [4 \pi ab + r_0 ^2 (10 \pi -13/3) - 4 \sqrt{2} \pi r_0 (a+b)]}
}
\end{equation}

Siendo con estas dos expresiones las alturas del hipercilíndro interior equivalente
$A=\sqrt{2} a z$,  $B=\sqrt{2} b z$. Recuperamos $z=1$ si las cuñas son ignoradas.

Una vez más, el cilindro exterior tendrá las mismas alturas que el interior,
pero el radio estará determinado por el volumen total:

\begin{equation}
R_{ext}=\frac{ab}{2\pi z^2}.
\end{equation}

\end{document}
