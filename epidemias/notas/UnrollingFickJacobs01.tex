\documentclass[letterpaperr,12pt]{article}

\usepackage[utf8]{inputenc}
\usepackage{amsmath,amssymb}

%\usepackage[spanish]{babel}

\usepackage[T1]{fontenc}  
%\usepackage{textcomp} Hey David, this is obsolete


\usepackage[cmbraces]{newtxmath}

\usepackage{xspace}
\usepackage{graphicx}

\usepackage{float}
\usepackage{caption}
\usepackage{subcaption}
%\usepackage{anysize}
\usepackage[margin=2cm]{geometry}

%\usepackage{bm}

\bibliographystyle{alpha}

\newcommand{\Acase}{\textbf{A}\xspace}
\newcommand{\Bcase}{\textbf{B}\xspace}
\newcommand{\Ccase}{\textbf{C}\xspace}
\newcommand{\Dcase}{\textbf{D}\xspace}
\newcommand{\Uno}{\textbf{1}\xspace}
\newcommand{\Dos}{\textbf{2}\xspace}


\newcommand{\defeq}{:=}
\newcommand{\mean}[1]{\left \langle #1 \right \rangle}
\newcommand{\rd}{\!\mathrm{d}}
\newcommand{\RR}{\mathbb{R}}
\newcommand{\vv}{\mathbf{v}}
\newcommand{\indicator}[1]{\mathbf{1}_{ \{   #1 \} } } 
\newcommand{\etal}{\emph{et al.\ }} 


%\renewcommand{\rmdefault}{ugm}
  

\title{Unrolling Fick-Jacobs}
\author{W. P. K. Zapfe}




\begin{document}

\maketitle


\section{The Story so far}

The application of Szabo's formula to the problem of
the \emph{contagious disks}  assumed that it was applicable
to consider the avaible free space as essentially equivalent to
the same space distributed inside more simmetrical shapes. 
To be precisse, we were trying to approximate an hypercube with
a fourdimensional absorving rod placed inside it by 
an anular section and two irrelevant ``heights''. We
have trustworthy arguments to support the idea that if this
approximation would have been the real system under study, 
Szabo's, David and Luca Formulas would have applied without
any significant modification. 

To apply Szabo's two dimensional formula, we needed an equivalent
anular geometry. Our approach was to obtain interior and exterior
radii by preserving the aviable free space and absorving surface. 
The extra dimensional ``heights'' of the cilinder turn out to be
unimportant. This approach lead us to an effective interior radious
as function of the disks original radii. Due to the
diagonal placement of this absorving object inside the hypercube,
the formula had to account for pointy tips, which were more
relevant as the radious became larger.  The rather unconfortable
formula accounted correctly for all these geometric conundrums. 

Once in possesion of an effective inner and outer radious in function
of the disks radius ($\rho(r)$) we could apply Szabo's Result. 
It shows a rather systematic discrepance with our numerical results. 
Being that these Numerical Simulations where done by two different
persons using diferent set of dynamic rules with different programing
styles and  languages, and that these coincide rather well, we are
lead to believe that these Numerics represent adecuatelly the 
\emph{mean encounter time} of the two disks, and that our reasoning
for the application of Szabo's is wrong, or is simply not
applicable.

The effective inner radius of the annulus is given in the formula below,
an the figure \ref{KDS01} shows our simulations (David, Karel) 
against Szabo's Formula. 

\begin{figure}[h]
  \centering
  \includegraphics[width=0.9\textwidth]{KarelDavidSzabo01.pdf}
  \caption{``Celdas Discretas'' is Karel's simulation, ``Vuelo Browniano'' is
    David's simulation.}\label{KDS01}
\end{figure}

We could make an intermediary approximation to the problem.
In this approximation we still shall use the rotational symetry, that
is, we shall still have sections with anular free space. But we shall preserve
from the original problems the ``narrowing'' of the free space along
the axis of the absorving cilinder. This gives us  
conic anular approximation. This is easy to represent in the 3d case, and
a bit more complicated in the 4d case, but nonetheless is the same
geometry. In this case Szabo's approximation would not be valid as it is, 
due to the narrowing of the free space. In order to test our conjecture
first we present simulations in this geometry and compare it to 
the ``animals in a territory'' simulations. 
Numerical simulations with the 4D conical approach seem to work well,
as the next figure (\ref{KDS02}) shows.

\begin{figure}[h]
  \centering
  \includegraphics[width=0.9\textwidth]{KarelDavidSzaboConos02.pdf}
  \caption{``Celdas Discretas'' is Karel's simulation, ``Vuelo Browniano'' is
    David's simulation.}\label{KDS02}
\end{figure}

We may try to use a Fick-Jacobs-Zwanzig-Rub-Reguera approach here.

\section{Backwards Formula for a Tube}

We have the backwards formula for the mean time encounter
with some \emph{exit}:
\begin{equation}\label{backwarde}
\exp(\beta U(x))
\frac{\rd}{\rd x}
\biggr[\exp (-\beta U(x)) \tilde{D}(x) \frac{\rd \tau}{\rd x} \biggr]=-1 
\end{equation}
The \emph{entropic potential} $U(x)$ is equal to  $-kT \ln (A(x)/A_0)$,
where we choose $A_0$ to make it zero at a convenient place. 
$A(x)$ is the cross section area orthogonal to the direction of the $x$ axis.

Then we also have an \emph{effective diffusion constant} $\tilde{D}(x)$,
that accorfing to Zwanzig can be approached by this expression:
\begin{equation}\label{DZ}
D_Z(x)=\frac{D}{1+[r'(x)]^2/2},
\end{equation}
and according to Rub and Reguera
can be approached by this expression:
\begin{equation}\label{DZ}
D_{RR}(x)=D(1-[r'(x)]^2/2).
\end{equation}
Now, $r(x)$ is the radius of a circle which has area $A(x)$, because they
made the approach on a tube with cilindrical symmetry and the targets at the
extremes of such tube. $r(x)$ has to be a smooth enough function.


\section{Testing the applicability}

First, we are going to \emph{``unroll''} the cylinder, a geometry
in which we allready known the formula for the mean first encounter 
with an inner cylindrical core. If the approach of unrolling the formula
is adecuate, then we could do something about it. It is important
to notice that we make no assumption about details on
the reflexion at the borders, and these could be a source of discrepancy.

Our ``tube'' goes along the radial direction, and its cross section
would be then the area of a perimetral sheet of the cone. We are disregarding
all symmetry in favour of preserving the cross section areas, see
figure \ref{diagramacil01}.

\begin{figure}
\centering
\includegraphics[width=0.25\textwidth]{CilindroConCilindroFJ.pdf}
\caption{The testing geometry.}\label{diagramacil01}
\end{figure}
So, we shall ``unroll'' the cylinder and consider the radial axis $\rho$
as the axis of our conceptual tube. The cross section area would be then
a sheet of the cylinder:
\begin{equation}
A(\rho)=2 \pi h \rho.
\end{equation}
As the absorving target is another cylinder with radius $\rho_0$, I 
shall use the entropìc potential in this manner:
\begin{equation}
U(\rho)=-kT \ln(\rho/\rho_0)
\end{equation}
The virtual radius of the cross section and its derivative are:
\begin{equation}
r(\rho)=\sqrt{2h\rho}, \;\; r'(\rho)=\sqrt{\frac{h}{r\rho}}.
\end{equation}
Putting these on the backwards equation yields
\begin{equation}\label{backwardeval}
\frac{\rho_0}{\rho}
\frac{\rd}{\rd \rho}
\biggr[\frac{\rho}{\rho_0} \tilde{D}(\rho) \frac{\rd \tau}{\rd \rho} \biggr]=-1 
\end{equation}
A first integration then gives:
\begin{equation}
\biggr[\frac{\rho}{\rho_0} \tilde{D}(\rho) \frac{\rd \tau}{\rd \rho} \biggr]=
-\frac{\rho^2}{2\rho_0}+\rho_0/2 
\end{equation}
or better
\begin{equation}
\int_0^{\tau_f} \rd \tau= 
\int_{\rho_0}^{\rho_f} \rd \rho 
\frac{1}{2 \tilde{D}(\rho)} \biggr( \frac{\rho_0^2}{\rho}-\rho \biggl)
\end{equation}

Now we choose either the Zwanzig or the Rub-Reguera forms for the 
effective difussion constant. 

\subsection{Using Zwanzig}

\begin{equation}
D \tau(\rho_f)= 
\int_{\rho_0}^{\rho_f} \rd \rho \biggl[ 
\frac{\rho_0^2}{2\rho}-\frac{\rho}{2}
+\frac{ h \rho_0^2}{8\rho^2}-\frac{h}{8}\biggr]
\end{equation}
yields
\begin{equation}
D \tau(\rho)=
\rho_0^2\ln(\rho/\rho_0)-1/4(\rho^2-\rho_0^2)
-h/8(\frac{\rho_0^2}{\rho}+\rho-2\rho_0)
\end{equation}
Which giveth zero when it should.

\subsection{Using Rub-Reguera}

\begin{equation}
D \tau(\rho_f)= 
\int_{\rho_0}^{\rho_f} \rd \rho \biggl[ 
\frac{2(\rho_0^2 -\rho^2)}{4\rho-h} \biggr]
\end{equation}
According to Wolfram Research this is:
\begin{equation}
D \tau(\rho)=
\frac{1}{64}\biggl[
(16\rho_0^2-h^2) \ln \biggl(\frac{4\rho-h}{4\rho_0-h}\biggr)
-4(2\rho^2+\rho h -2 \rho \rho_0 -\rho_0 h) 
\biggr]
\end{equation}

Now lets check this out...

\end{document}

