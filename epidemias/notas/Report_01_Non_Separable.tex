\documentclass[letterpaperr,12pt]{article}

\usepackage[utf8]{inputenc}
\usepackage{amsmath,amssymb}


\usepackage[T1]{fontenc}  
\usepackage{textcomp}


%\usepackage[garamond,cmbraces]{newtxmath}

\usepackage{xspace}
\usepackage{graphicx}

\usepackage{float}
\usepackage{caption}
\usepackage{subcaption}
%\usepackage{anysize}
\usepackage[margin=2cm]{geometry}

%\usepackage{bm}

\bibliographystyle{alpha}

\newcommand{\Acase}{\textbf{A}\xspace}
\newcommand{\Bcase}{\textbf{B}\xspace}
\newcommand{\Ccase}{\textbf{C}\xspace}
\newcommand{\Dcase}{\textbf{D}\xspace}
\newcommand{\Uno}{\textbf{1}\xspace}
\newcommand{\Dos}{\textbf{2}\xspace}

\renewcommand{\rmdefault}{ugm}
  

\title{Two territorial ``Animals'' encounter time.}
\author{W. P. K. Zapfe}


\begin{document}

%r\marginsize{2cm}{2cm>}{2cm}{2cm}
\maketitle

\begin{abstract}

I will pose the considerations and difficulties in extending the 
treatment in the problem of contagious territorial animals for one dimensional 
random walker models treated in \cite{LucaDavid} to a more
realistic 2D random walker model. 

\end{abstract}


\section{Four apparently simple cases}

We shall distinguish four variants of the problem. In all
of them the animal labelled as \Uno has a square territory
with sides size 2, whose coordinates in orthogonal directions
range from -1 to 1. The second animal, labelled \Dos, has
different territorial freedom in each case.
Horizontal or West to East directions will have
coordinates indicated by $x$, vertical or South to North, by
$y$, as usual. 

In the case \Acase, \Dos is shy and precautions. It only
enters the territory of \Uno along a narrow strip of
length $1-l$, which we can model as
a straight line. The case \Bcase is a slight modification of 
this case, in which \Dos is able to cross \Uno territory 
from side to side. In the case \Ccase, \Dos is able to move
along a two dimensional strip on the eastern side of \Uno's
territory. Finally, \Dcase is simply a representation
of the generic intersection of two square domains.
The four cases are diagrammatically illustrated in the 
 figure \ref{fig::casos}. 

\begin{figure}[h]
  \centering
  \begin{subfigure}[b]{0.45\textwidth}
    \centering
    \includegraphics[width=\textwidth]{CasoTerritorioA.pdf}
    \caption{\Acase}
    \label{figAcase}
  \end{subfigure}
  \begin{subfigure}[b]{0.45\textwidth}
    \centering
    \includegraphics[width=\textwidth]{CasoTerritorioB.pdf}
    \caption{\Bcase}
    \label{figBcase}
  \end{subfigure}

  \begin{subfigure}[b]{0.45\textwidth}
    \centering
    \includegraphics[width=\textwidth]{CasoTerritorioC.pdf}
    \caption{\Ccase}
    \label{figCcase}
  \end{subfigure}
  \begin{subfigure}[b]{0.45\textwidth}
    \centering
    \includegraphics[width=\textwidth]{CasoTerritorioD.pdf}
    \caption{\Dcase}
    \label{figDcase}
  \end{subfigure}
  \caption{The four different variants}\label{fig::casos}
\end{figure}

\section{Random Walker considerations}

It is known that a coarse grain approximation to the movement of an 
animal inside its territory is a random walker. We start
our reasoning by dividing the territory of both animals into equal 
square cells. Then we will think of the two animals as a single
random walker moving on a space whose dimension is the sum of 
the dimensions of both territories. 
For case \Acase and \Bcase this will be a cube, on the other
two it will become a teseract, as the animals have one more
degree of freedom to explore. A schematic representation
of the \Acase is on figure \ref{CeldasA}. The sets of cells
where encounter does occur is marked by a blue colour outline,
and a sphere represents an occupied cell, that is, the
animals are on the coordinates inside their territory.

\begin{figure}[h]
  \centering
  \begin{subfigure}[b]{0.49\textwidth}
    \centering
    \includegraphics[width=0.99\textwidth]{Celdas_7_con_encuentro.pdf}
    \caption{7 cells by side.}
    \label{gruesaceldas}
  \end{subfigure}
  \begin{subfigure}[b]{0.49\textwidth}
    \centering
    \includegraphics[width=0.99\textwidth]{Celdas_20_con_encuentro.pdf}
    \caption{20 cells by side.}
    \label{finaceldas}
  \end{subfigure}
  \caption{The variation \Acase of the problem with 
    discrete random walking, represented on a single
    3D cube.}\label{CeldasA}
\end{figure}

The conditions for encounter are as follows:

\begin{align}
x_1&=x_2+l\\
y_1&=y_2. \text{ where  $y_2$ is a constant.}
\end{align}

Now a problem should become apparent. As we get to the
continuous limit, the cells become smaller and smaller.
The set of encounter cells reduces to a line, which the
random walker has probability zero of encountering.
In order to solve this we consider that the animals
have a radius of interaction of size $\epsilon$, where
they can get sickness from each other. Still in a discrete
cell approximation this would look as in the figure \ref{celdasradio}.

\begin{figure}[h]
  \centering
  \begin{subfigure}[b]{0.49\textwidth}
    \includegraphics[width=0.99\textwidth]{Celdas_15_con_radio.pdf}
    \caption{Discrete version.}
    \label{celdasradio}
  \end{subfigure}
  \begin{subfigure}[b]{0.49\textwidth}
    \includegraphics[width=0.99\textwidth]{CilindroDescentrado.pdf}
    \caption{continuous version.}
    \label{continuoradio}
  \end{subfigure}
    \caption{This time, we set for the encounter a radius of
    $\epsilon$ for each animal. In the discrete cell approximation, 
    this thickens the set of available cells, here showed as a lighter blue.
    The limit is a cylinder.}
    \label{conradio}
\end{figure}

The expression for the encounter, taken into account that
the radius of interaction is different from zero, turns this way:

\begin{align}\label{radii}
 (x_1-x_2-l)^2+(y_1-y_2)^2 \leq (2\epsilon)^2 
\end{align}

In the continuous limit, this will draw a cylinder laying
diagonally on the full phase space, as seen on the figure 
\ref{continuoradio}. The case \Bcase is the limit for which $l=0$,
and then the cylinder goes from one edge to the opposite. In
such a geometry, it appears to be easy to threat the continuous limit
in a volume preserving smooth approximation. We could
try to maximise the symmetries by replacing the cube by a sphere or
a cylinder. This second option turns out to be inconvenient
as it doesn't reflect the fact that in case \Acase the
volume of the inner cylinder decreases as the penetration of
\Dos into \Uno territory. Still, this would be
the only case amenable to treatment by a known
separation of variables.

\begin{figure}[h]
  \centering
  \begin{subfigure}[b]{0.29\textwidth}
    \includegraphics[width=0.79\textwidth]{CilindroDiagonal01.png}
    \caption{}
    \label{CilDiag}
  \end{subfigure}
  \begin{subfigure}[b]{0.29\textwidth}
    \includegraphics[width=0.99\textwidth]{EsferaCilindro.png}
    \caption{}
    \label{EsferaCil}
  \end{subfigure}
  \begin{subfigure}[b]{0.29\textwidth}
    \includegraphics[width=0.99\textwidth]{CilindroenCilindro01.png}
    \caption{}
    \label{CilCil}
  \end{subfigure}
    \caption{In the case \Bcase is easier to approximate the 
    cube with a symmetric solid, using the axis of the cylinder
    as the rotation axis. In any other case this gives misleading figures.}
    \label{smoothapprox}
\end{figure}

It is the approximation shown in figure \ref{CilCil} the one where we could
proceed to use a continuous diffusive treatment and use separation of
variables to solve the equation. It comes with severe limitations, though,
as this geometry does not take into account the narrowing of the space
in the extreme sides of the absorbing inner cylinder. 

Of course the convenient cylindrical coordinates are in the scaled
space where the equation for the inner absorbing surface reads

\begin{equation}
x^2+y^2 = 2 \epsilon^2,
\end{equation}

so that we do not have to worry about elliptical cylinders.
That is achieved by the simple $pi/4$ rotation,  
\begin{equation}\label{rotacion}
  \begin{pmatrix}
    a \\
    A
  \end{pmatrix}
  =\frac{1}{\sqrt{2}}
  \begin{pmatrix}
    1 & -1 \\
    1 & 1
  \end{pmatrix}
  \begin{pmatrix}
    a_1 \\
    a_2
  \end{pmatrix},
\end{equation}

where we substitute the $a$ by $x$ and $y$. The fact that we take into
consideration the $y_2$ coordinate even when it is a constant help us
to deal with unsavoury $\sqrt{2}$ factors and pave the way for the 
case \Dcase.

It is also easier to describe a sphere in cylindrical coordinates than the 
other way around, so we stick with this for considering the case depicted
in the figure \ref{EsferaCil}. The $X$ axis according to the transformation
\ref{rotacion} will be our $z$ axis in cylindrical coordinates.

\begin{align}
x=&\rho\cos\theta \\
y=&\rho\sin\theta \\
X=z.
\end{align}

The diffusive equation to solve has a relative simple expression in such a system:
\begin{equation}\label{diffcil}
  \partial_t f = D \Big(\frac{1}{\rho} \frac{\partial}{\partial\rho}\rho\frac{\partial}{\partial\rho}f
  +\frac{1}{\rho^2}\frac{\partial^2}{\partial \theta^2}f
  +\frac{\partial^2}{\partial z^2}f\Big).
\end{equation}

The boundary conditions are as follows: for the outer border, the boundary reflects the
walk, meaning that the animals who reach the limit of their territory turn around.
The inner border is an absorbing boundary, thereby signifying that an encounter
has occurred. The mathematical expressions for these conditions are as follows.

\begin{align}
\nabla f \cdot \hat{n} &=0 \text{ at the exterior border} \\
f &=0 \text{ at the inner border}.
\end{align}

In the case that we approximate the available phase space with a cylinder, 
the normal vector becomes the radial vector and the separation of variables
looks feasible. In the case of a spherical outer border this
condition looks a bit more complex in the set of coordinates chosen.

\begin{equation}
\frac{1}{\rho}\frac{\partial}{\partial\rho} f +
\frac{1}{z}\frac{\partial}{\partial z} f =0 \text{ when } z^2+\rho^2 = R_0 ^2.
\end{equation}

This treatment is far from a generic case, and both approximations lead 
to terrible expressions for the more general \Acase case. We could use an unfolding trick on
the space shown in the figure \ref{continuoradio}. We make reflections on the 
square side who has  contact with the cylinder, producing the toroidal inner border
shown in figure \ref{Unfold}. The sensible approximation is to consider this
figure as annular cylinder and use cylindrical coordinates with rotation axis parallel
the $y_2$ coordinate axis. This is shown in \ref{ToroChato}.


\begin{figure}[h]
  \centering
  \begin{subfigure}[b]{0.29\textwidth}
    \includegraphics[width=0.99\textwidth]{ToroFeo01.png}
    \caption{Unfolding}
    \label{Unfold}
  \end{subfigure}
  \begin{subfigure}[b]{0.29\textwidth}
    \includegraphics[width=0.99\textwidth]{ToroChato01.png}
    \caption{Smoothing}
    \label{ToroChato}
  \end{subfigure}
    \caption{In the case \Bcase is easier to approximate the 
    cube with a symmetric solid, using the axis of the cylinder
    as the rotation axis. In any other case this gives misleading figures.}
    \label{UnfoldandSmooth}
\end{figure}

Maybe this could be amenable by a dissection of the separation of variables.
We should first treat the distribution of passage times from the part of the
cylinder which doesn't contain an absorbing ring to that section. Then there we should
divide it into three different events: 
where the particle went directly to the absorbing ring, where it went to the
inside, and where it went to the outer section. The last two produce a further complication.
The particle can return to the free space above and below the ring. So we can have a 
quite complicate chain of events to calculate the first encounter time \ldots



\end{document}

